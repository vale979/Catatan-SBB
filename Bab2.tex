\chapter{Sistem Bilangan Bulat}
	Bahasan struktur bilangan bulat akan dimulai dari hal yang paling familiar dengan kita, yaitu sistem bilangan bulat beserta sifat-sifatnya.
	\section{Bilangan Bulat}
	Tinjau sistem matematika $(\mathbb{Z},+,\times)$ dengan $+$ dan $\times$ adalah operasi penjumlahan dan perkalian biasa. Kita punya sifat-sifat berikut untuk setiap $a,b,c \in \mathbb{Z}$:
	\begin{enumerate}
		\item Sifat asosiatif, yakni $(a+b)+c = a+(b+c)$ dan $a \times (b \times c) = (a \times b) \times c$
		\item Sifat komutatif, yakni $a+b = b+a$ dan $a \times b = b \times a$
		\item Ada unsur identitas $e \in \mathbb{Z}$ sehingga $ae = ea = a$. Untuk penjumlahan $e = 0$ sehingga $a + 0 = 0 + a = 0$ dan untuk perkalian $e = 1$ sehingga $a \times 1 = 1 \times a = a$
		\item Ada unsur invers penjumlahan $-a$ sehingga $a + (-a) = (-a) + a = 0$
		\item Sifat distributif, yakni $a \times (b + c) = a\times b + a \times c$
	\end{enumerate}
	Untuk setiap sistem matematika yang memenuhi sifat-sifat tersebut, kita sebut sebagai \textbf{gelanggang} (\textit{ring}). Perhatikan pula bahwa pada gelanggang bilangan bulat, misalkan $a,b \in \mathbb{Z}$ dengan $ab = 0$. Haruslah $a = 0$ atau $b = 0$. Sifat ini tidak selalu berlaku pada sembarang gelanggang (misalnya pada perkalian matriks). Gelanggang yang memenuhi sifat tersebut kita sebut sebagai \textbf{daerah integral} (\textit{integral domain}).
	\\
	
	Berikutnya, kita akan buktikan hasil yang selama ini sudah kita kenal (tetapi sebenarnya bukanlah aksioma)
	
	\begin{theorem}
		Untuk $a,-a,b \in \mathbb{Z}$ berlaku
		\begin{itemize}
			\item $0a =0$
			\item $(-a)b = a(-b) = -(ab)$
		\end{itemize}
	\end{theorem}
	\begin{proof}
		\begin{enumerate}
			\item Dari definisi, $0 + 0 = 0$. Kita dapatkan
			\begin{equation*}
				\begin{split}
				0a & = (0+0)a\\
				& = 0a + 0a\\
				\end{split}
			\end{equation*}
			Menjumlahkan kedua ruas dengan $-(0a)$, kita dapatkan $0a = 0$.
		\item Dari sifat di atas, kita tahu bahwa $a + (-a) = 0$ dan $0b = 0$ sehingga $0 = (a + (-a))b = ab + (-a)b$ (sifat distributif). Dengan menjumlahkan kedua ruas dengan $-ab$ kita dapatkan
		\begin{equation*}
		\begin{split}
			-ab & = ab + a(-b) + (-ab)\\
			-ab & = a(-b) 
		\end{split}
		\end{equation*}
		Dengan cara serupa, $0 = (a + (-a))b = ab + a(-b)$ (perkalian bersifat komutatif) sehingga
		\begin{equation*}
			\begin{split}
				0 & = ab + a(-b)\\
				-ab & = -ab + ab + a(-b)\\
				-ab & = a(-b)
			\end{split}
		\end{equation*}
		Dengan demikian, $(-a)b = a(-b) = -ab$
		\end{enumerate}
	
	Jadi, teorema tersebut terbukti.
	\end{proof}
	
	Berikutnya, kita akan definisikan keterbagian di bilangan bulat. Misal $a,b \in \mathbb{Z}$, definisikan
	\begin{enumerate}
		\item Unsur $a$ dikatakan membagi $b$, notasikan sebagai $a | b$, jika ada $c \in \mathbb{Z}$ sehingga $b = ac$
		\item Unsur $a$ dikatakan unit jika ada $b$ bulat sehingga $ab = 1$. Dapat diperiksa bahwa unit di $\mathbb{Z}$ hanyalah -1 dan 1
		\item Nilai mutlak dari $a$ adalah $$ |a| =  
		\begin{cases}
			a &\mbox{untuk $a \ge 0$}\\
			-a &\mbox{untuk $a < 0$}
		\end{cases}	$$
	\end{enumerate}
	
	Kemudian kita akan bahas tentang algoritma pembagian. Hal ini akan sering digunakan ketika kita membahas pembagi sekutu terbesar.
	\begin{theorem}[Algoritma Pembagian]
		Misal $a,b \in \mathbb{Z}$ dengan $a \ne 0$. Ada $p,r \in \mathbb{Z}$ unik sehingga $$b = ap + r$$ dengan $0 \le r < |a|$
	\end{theorem}
	\begin{proof}
		Kita perlu tunjukkan dulu $p,r$ ada, lalu tunjukkan bahwa $p,r$ unik. Tinjau dahulu kasus $a > 0$. 
		Definisikan himpunan $S = \{ b - ka \ | \ b-ka \ge 0, k \in \mathbb{Z} \}$. Dari \textit{well-order} kita tahu bahwa $S$ punya unsur terkecil asal $S$ tak hampa. Ini akan kita buktikan dulu. Untuk $b \ge 0$ jelas $S$ tak hampa (tinjau $k = 0$). Untuk $b < 0$ tinjau $k = b$ sehingga $b - ka = b - ba = b(1-a)$. Karena $a > 0$ dan $a$ bulat pastilah $(1-a)$ nol atau negatif sehingga $b(1-a) \ge 0$. Jadi, $S$ tidak kosong. Dengan demikian, kita bisa menggunakan prinsip \textit{well-order} untuk mendapatkan $S$ punya unsur terkecil. Misalkan unsur terkecilnya adalah $$r = b - k'a$$ tentu $0 \le r < a$. Karena $r \in S$, $r$ tentu non-negatif. Cukup dibuktikan $r < a$. Misal sebaliknya, $r \ge a$, berarti $b - k'a \ge a \iff b - (k'+1)a \ge 0 \in S$. Berarti, $r$ bukan unsur terkecil sehingga terjadi kontradiksi dengan prinsip \textit{well-order}. Jadi, haruslah $r < a$. Untuk mencari $p$, mudah dilihat bahwa $p = k'$ sehingga ada $p,r$ yang membuat $b = ap + r$. (Untuk $a < 0$ serupa)
		\\
		
		Kemudian untuk menunjukkan $p,r$ unik, perlu ditunjukkan untuk $b = ap_1+r_1$ dan $b = ap_2 + r_2$ berlaku $p_1 = p_2$ dan $r_2 = r_1$. Tanpa mengurangi keumuman, asumsikan $r_1 \ge r_2$. Kita dapatkan $ap_1 + r_1 = ap_2 + r_2$ sehingga $r_1 - r_2 = a(p_2 - p_1)$. Jadi, $r_1 - r_2$ kelipatan $a$. Karena $0 \le r_1, r_2 < a$ dan $r_2 \ge r_1$, haruslah $0 \le r_2 - r_1 < a$. Satu-satunya kelipatan $a$ pada \textit{range} tersebut adalah $0$ sehingga $r_1 - r_2 = 0 \iff r_1 = r_2$. Akibatnya juga haruslah $p_1 = p_2$. Jadi, $p,r$ unik.
	\end{proof}
	
	Berikutnya, kita akan buktikan eksistensi generator atau pembangun bagi subhimpunan $\mathbb{Z}$.
	\begin{theorem}
		Untuk $S \subseteq \mathbb{Z}$ tak hampa dengan $a + b, a - b$ di $S$ untuk setiap $a,b$ di $S$, ada $n \in S$ sehingga $$S = \{kn \, | \, k \in \mathbb{Z}\}$$
	\end{theorem}
	\begin{proof}
		Ambil $a \in S$, jelas $a-a = 0 \in S$. Untuk $S = {0}$, kita bisa pilih $n = 0$ sehingga $S = \{k0 \, | \, k \in \mathbb{Z} \}$. Untuk kasus lainnya, misal $a \ne 0$, berarti $-a = 0 - a \in S$ sehingga $\{-a, 0, a\} \subseteq S$. Konsekuensinya, himpunan $S$ memuat bilangan positif. Definisikan himpunan $T = \{ t \, | \, t \in S, t > 0 \}$. Misalkan $x$ unsur terkecil di $T$. Untuk $k = 0$, $kx = 0$. Untuk $k > 0$, $kx = x + ... + x$ (sebanyak $k$ kali). Untuk $k < 0$, ada $m$ bulat positif sehingga $k = -m$ dan $kx = (-m)x = m(-x) = (-x) + ... + (-x)$ (sebanyak $m = -k$ kali). Akibatnya, $\{ kx \, | \, k \in \mathbb{Z} \} \subseteq S$. Ambil $y \in S$, ada $p,r \in \mathbb{Z}$ sehingga $y = px + r$ dengan $0 \le r \le x$. Jelas bahwa $r = y - px \in S$. Karena $x$ unsur terkecil di $S$ dan $0 \le r \le x$, haruslah $r = 0$. Dengan demikian, $y = px \in S$ sehingga $S = \{kn \, | \, k \in \mathbb{Z}\} $
	\end{proof}
	Definisi-definisi berikut akan membantu untuk bahasan subbab berikutnya.
	\\
	\begin{enumerate}
		\item Unsur $a,b \in \mathbb{Z}$ dikatakan \textbf{sekawan} jika $a|b$ dan $b|a$
		\item Unsur $p \in \mathbb{Z}$ bukan unit dikatakan prima jika $p | ab$ mengimplikasikan $p | a$ atau $p | b$
	\end{enumerate}
	\section{Pembagi Sekutu Terbesar}
	Tinjau dulu pembagi sekutu terbesar dari 2 bilangan bulat. Untuk $a,b,d \in \mathbb{Z}$. Unsur $d$ dikatakan pembagi sekutu terbesar bagi $a,b$ jika $d | a$ dan $d|b$ dan jika ada $c$ bulat yang menyebabkan $c | a$ dan $c | b$ maka $c | d$. Kita notasikan sebagai $(a,b) = d$. Beberapa referensi ada yang menuliskannya sebagai $GCD(a,b)$ (GCD di sini adalah \textit{Greatest Common Divisor}) ataupun $FPB(a,b)$. Notasi $()$ dipilih di sini untuk memudahkan saja.\\
	
	Kita definisikan juga untuk dua buah $a,b$ jika $(a,b) = 1$ kita katakan $a,b$ \textbf{relatif prima} atau \textbf{saling prima}.
	\\
	
	Perlu diperhatikan bahwa mendefinisikan GCD tidak serta-merta mengimplikasikan GCD ada. Kali ini kita akan buktikan eksistensinya.
	
	\begin{theorem}[Eksistensi GCD]
		Untuk $a,b \in \mathbb{Z}$ tidak keduanya nol ada $d \in \mathbb{Z}$ unik sehingga $d = (a,b)$ dan $d$ dapat ditulis sebagai $d = m_0a + n_0b$ untuk suatu $m_0,n_0 \in \mathbb{Z}$
	\end{theorem}
	
	\begin{proof}
		 Definisikan $A = \{ma + nb \ | \ m,n \in \mathbb{Z}\}$. karena $a,b$ tak keduanya nol, $A$ punya unsur tak nol. Jika $x \in A$ dengan $x < 0$, maka $-x = (-m_1)a + (-n_1)b \in A$ untuk suatu $m_1, n_1$ bulat. Jadi, $A$ punya unsur positif. Dari \textit{well-order}, $A$ punya unsur positif terkecil, sebut saja $c$. Karena $c \in A$, ada $m_0, n_0$ sehingga $m_0a + n_0b = c$. Klaim: $c$ adalah GCD dari $a,b$. Perhatikan bahwa jika $d | a$ dan $d |b$, haruslah $d | (m_0 a + n_0b)$ sehingga $d|c$. Dengan algoritma pembagian, dapat ditunjukkan bahwa $c | a$ dan $c | b$. Karena $c$ unsur positif terkecil di $A$, dan $c | a$ dan $c | b$, haruslah $c = (a,b)$, yaitu GCD dari $a,b$. Jadi, GCD ada.
		 \\
		 
		 Untuk membuktikan GCD unik, misal ada $t > 0$ sehingga $t$ juga memenuhi definisi GCD (selain $c$). Haruslah $t | c$ dan $c | t$, sehingga $c = \pm t$. Namun, $c$ dan $t$ keduanya positif, sehingga $c = t$. Jadi, GCD unik.
	\end{proof}
	
	Kemudian tinjau untuk lebih dari $2$ unsur. Misalkan $\{ a_1, a_2, ..., a_n \} \subseteq \mathbb{Z}$ dengan $a_i \ne 0$. Definisikan unsur $d \in \mathbb{Z}$ sebagai pembagi sekutu terbesar bagi $a_1, a_2, ..., a_n$ jika
	\begin{itemize}
		\item $d | a_i$ untuk setiap $i \in {1,2,...,n}$
		\item Jika ada $c | a_i$ untuk setiap $i \in {1,2,...,n}$, maka $c | d$
	\end{itemize}
	Kita notasikan $(a_1, a_2, ..., a_n) = d$.
	\\
	
	\textbf{Diskusi. }Coba anda buktikan bahwa GCD untuk $n$ unsur ada.
	\section{Bilangan Prima dan Faktorisasi}
	Kita awali dahulu bahasan pada subbab ini dengan membahas bilangan prima. Suatu bilangan $p$ (untuk sekarang asumsikan $p \in \mathbb{Z}$) dikatakan \textbf{prima} jika $p | ab$ mengakibatkan $p | a$ atau $p | b$ dan $p$ bukan unit (sehingga di bilangan bulat, $1$ dan $-1$ bukanlah prima).\\
	
	Di sekolah dasar dan menengah, kita sudah cukup familiar dengan konsep faktorisasi prima dari suatu bilangan. Namun, dari mana kita yakin bahwa setiap bilangan bulat dapat difaktorkan menjadi faktor-faktor prima (anggap satu dan nol sebagai perkalian kosong)? Kita tinjau dahulu kasus untuk bilangan bulat positif lebih dari 1. Untuk itu, kita punyai teorema berikut:
	\begin{theorem}[Teorema Dasar Aritmatika]
		Setiap bilangan bulat lebih dari 1 adalah bilangan prima atau merupakan hasil perkalian bilangan prima secara tunggal
	\end{theorem}
	Untuk membuktikan teorema ini, ada dua hal yang perlu diperhatikan. Pertama, menunjukkan bahwa faktorisasinya ada, kemudian menunjukkan bahwa faktorisasi tersebut tunggal. Dengan kata lain, apabila suatu bilangan $x \in \mathbb{Z}$ punya lebih dari satu cara untuk difaktorkan sedemikian sehingga $x = p_1 p_2 ... p_n = q_1 q_2 ... q_m$, haruslah berlaku $m = n$ dan ada suatu permutasi $\sigma : [1,n] \rightarrow [1,n]$ sehingga $p_i = q_{\sigma{(i)}}$ (perbedaan hanya pada urutan perkalian saja).
	\begin{proof}
		Kita mulai buktinya dengan membuktikan eksistensi faktor prima. Tinjau $n = 2$, karena 2 prima, jelas 2 dapat ditulis sebagai perkalian faktor-faktor prima (yakni dirinya sendiri). Andaikan untuk setiap $n < k$ benar bahwa setiap bilangan bulat positif lebih dari 1 punya faktor pirma. Untuk $k$ prima, jelas. Untuk $k$ tidak prima, haruslah ada $x,y$ dengan $1 < x \le y < k$ dengan $k = xy$. Dari asumsi kita sebelumnya, $x$ dan $y$ dapat ditulis sebagai perkalian faktor-faktor prima. Akibatnya, $k$ juga dapat ditulis sebagai perkalian faktor-faktor prima. Dari prinsip induksi kuat, kita simpulkan setiap bilangan bulat lebih dari 1 dapat ditulis sebagai perkalian faktor-faktor prima.\\
		
		Berikutnya, kita akan tinjau ketunggalannya. Andaikan ada suatu bilangan yang punya lebih dari 1 cara untuk memfaktorkannya, dari prinsip \textit{well-ordering} haruslah ada bilangan terkecil yang demikian, sebut saja $x$. Misalkan $x = p_1 p_2 ... p_n = q_1 q_2 ... q_m$. Akibatnya, $p_1 | q_1 q_2 ... q_m$. Karena $p_1$ prima, ada $i \in [1,m]$ sehingga $p_1 | q_i$. Tanpa mengurangi keumuman, misalkan $i = 1$ sehingga $p_1 | q_1$. Karena keduanya prima, haruslah $p_1 = q_1$. Kita dapatkan $p_2 p_3 ... p_n = q_2 q_3 ... q_m < x$. Kontradiksi dengan fakta bahwa $x$ adalah bilangan terkecil yang punya lebih dari satu cara untuk difaktorisasi prima. Kita simpulkan bahwa faktorisasi prima tersebut haruslah tunggal.
	\end{proof}
	Ternyata faktorisasi tunggal tidak selalu berlaku di setiap sistem matematika. Sebagai contoh, jika kita meninjau sistem matematika yang dibangun oleh himpunan $\{ a+b\sqrt{-5} \, | \, a,b \in \mathbb{Z} \}$, kita dapat lakukan faktorisasi $6 = 2 \times 3 = (1+\sqrt{-5})(1-\sqrt{-5})$. Lebih lanjut tentang ini akan kita bahas ketika kita mengkaji daerah faktorisasi tunggal di bab 4.\\
	
	Berikutnya, kita akan tinjau akibat dari teorema sebelumnya. Hasil ini dibahas pertama kali pada buku \textit{Elements} karya Euclid.
	\begin{theorem}
		Ada tak hingga banyaknya bilangan prima
	\end{theorem}
	\begin{proof}
		Andaikan ada berhingga bilangan prima, definisikan himpunan $P = \{ p_1, p_2, ..., p_n \}$ sebagai himpunan semua bilangan prima. Berikutnya, tinjau $k = p_1 p_2 ... p_n + 1$. Jelas $k > 1$. Dari teorema dasar aritmatika, $k$ punya faktor prima, sebut saja $p$. Andaikan ada $i \in [1,n]$ sehingga $p = p_i$, haruslah $p|1$. Akibatnya, haruslah $p \ne p_i$ untuk setiap $i \in [1,n]$. Jadi, ada $p$ prima dengan $p \not\in P$, himpunan semua bilangan prima. Kontradiksi. Akibatnya, haruslah ada tak hingga banyaknya bilangan prima.
	\end{proof}
	