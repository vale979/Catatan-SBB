\documentclass[11pt,b5paper]{book}
\usepackage[utf8]{inputenc}
\usepackage{amsmath}
\usepackage{amsfonts}
\usepackage{amsthm}		%theorem
\usepackage{amssymb}
\usepackage{graphicx}
\usepackage{tikz-cd}	%commutative diagram
\usepackage{adjustbox}

\author{Valerian Mahdi Pratama}
\title{Catatan Struktur Bilangan Bulat}
\date{}

\setlength\parindent{0pt}

%define utk amsthm
\newtheorem{theorem}{Teorema}[section]
\newtheorem{corollary}{Akibat}[theorem]
\newtheorem{inneraxiom}{Aksioma}

%rename beberapa nama default
\renewcommand*{\proofname}{Bukti}
\renewcommand{\chaptername}{Bab}
\renewcommand{\contentsname}{Daftar Isi}
\renewcommand\appendixname{Lampiran}


\begin{document}

\maketitle

%Hitungan utk numbering table of content
\setcounter{tocdepth}{1} 

\tableofcontents

\frontmatter
\chapter{Pengantar Untuk Versi Pertama}
	Catatan ini dibuat sebagai penunjang dan referensi tambahan ringkas untuk materi kuliah struktur bilangan bulat. Sebagian besar isi dari catatan ini adalah digitalisasi dari catatan penulis saat mengikuti kuliah struktur bilangan bulat di prodi Matematika ITB pada semester genap tahun akademik 2018-2019. Beberapa tambahan yang relevan sebagian besar diambil dari buku \textit{Aljabar} karya Achmad Ariffin, buku \textit{Abstract Algebra} karya I.N. Herstein, atau hasil corat-coret penulis. Kedua buku tersebut merupakan buku teks yang dijadikan referensi pada kuliah struktur bilangan bulat. Sebagai apresiasi saya terhadap prinsip kemerdekaan atas ilmu pengetahuan, catatan ini bebas untuk disebarluaskan, dicetak, dan digunakan untuk tujuan non-komersil.
	\\
	
	Sesuai namanya, catatan ini hendaknya hanya digunakan sebagai catatan untuk \textit{review} materi terkait. Catatan ini tidak ditujukan sebagai pengganti buku teks, apalagi referensi satu-satunya untuk mempelajari struktur bilangan bulat. Sebaiknya pembaca terlebih dahulu membaca buku teks yang menjadi referensi utama sebelum membaca catatan ini. Karena kuliah ini merupakan kuliah tingkat dua, penulis mengasumsikan pembaca sudah familiar dengan notasi-notasi yang umum digunakan pada operasi himpunan dan pemetaan (notasi penting lainnya akan diberikan definisinya). Penulis juga mengasumsikan pembaca telah familiar dengan metode penulisan bukti yang diajarkan pada kuliah pengantar matematika. Istilah grup sengaja penulis gunakan meskipun tidak masuk dalam kajian kuliah struktur bilangan bulat untuk mempermudah pendefinisian beberapa struktur. Sekilas bahasan tentang grup dapat diperiksa pada bagian lampiran.
	\\
	
	Meskipun ini bukanlah buku teks, penulis berharap metode membaca catatan ini serupa dengan membaca buku teks. Pembaca harus aktif memverifikasi hasil yang didapat pada catatan ini dengan setidaknya selembar kertas dan sebuah pulpen serta mempertanyakan motivasi dibalik pendefinisian suatu ide (umumnya jawabannya akan anda dapatkan setelah membaca topik bahasan terkait ide tersebut). Beberapa soal (sebagian diambil dari soal yang dibahas di kelas) disisipkan penulis untuk menambah gambaran sekaligus melatih konsep yang dibahas.
	\\
	
	\textit{Feedback} untuk catatan ini dapat diberikan dengan menghubungi penulis via surel di valerianmp@valmc2.com atau melalui sosial media lainnya. Jika dirasa perlu, versi revisi berikutnya akan dirilis untuk memperbaiki catatan ini. Sebagai akhir dari pengantar ini, tak lupa penulis mengucapkan terima kasih kepada Bu Dellavitha sebagai dosen pengampu kuliah struktur bilangan bulat beserta teman-teman sekelas dalam kuliah struktur bilangan bulat pada semester genap 2018-2019. Penulis berharap catatan ini dapat dimanfaatkan sebaik-baiknya, entah dalam membantu pada kuliah struktur bilangan bulat atau hal lainnya yang relevan.
	\\
	\begin{flushright}	
	Untuk umat manusia yang lebih baik,\\
	\hfill \break
	Valerian Mahdi Pratama
	\end{flushright}
		
\chapter{Notasi}
	Beberapa notasi yang digunakan pada catatan ini
	\begin{itemize}
		\item $\mathbb{Z}$: himpunan bilangan bulat
		\item $\mathbb{N}$: himpunan bilangan asli (blangan bulat positif yang lebih besar dari nol)
		\item $\mathbb{R}$: himpunan bilangan real
		\item $\mathbb{Q}$: himpunan bilangan rasional
		\item $\emptyset$: himpunan kosong
		\item $\{x \, | \, x \in S \}$: himpunan semua $x$ dengan $x$ anggota $S$
		\item $a \implies b$: jika $a$ maka $b$; $a$ mengakibatkan $b$
		\item $a \iff b$: $a$ jika dan hanya jika $b$
		\item $a \in S$: $a$ adalah anggota himpunan $S$
		\item $a \notin S$: $a$ bukan anggota himpunan $S$
		\item $H \subseteq S$: $H$ subhimpunan bagi $S$
		\item $H \subset S$: $H$ subhimpunan sejati bagi $S$
		\item $\forall a$: untuk setiap $a$
		\item $\exists a$: terdapat $a$
		\item $S \rightarrow S'$: (pemetaan) dari $S$ ke $S'$
		\item $a \mapsto a'$: $a$ dipetakan ke $a'$
		\item $\kappa(a)$: kelas ekuivalen yang dibangun oleh $a$ (untuk suatu relasi ekuivalen)
		\item $(a,b)$: tupel; pembagi sekutu terbesar dari $a$ dan $b$ (tergantung konteks)
		\item $ a | b$: $a$ membagi $b$
		\item $a \not| \ b$: $a$ tidak membagi $b$
		\item $F[x]$: lapangan polinomial dengan koefisien dari lapangan $F$
		\item $\mathbb{Z}_{\ge 0}$: Himpunan bilangan bulat nonnegatif
		\item $\langle x_1, x_2, ..., x_n \rangle$: Himpunan yang dibangun oleh himpunan  $\{x_1, x_2, ..., x_n\}$ (perhatikan konteksnya)
	\end{itemize}
	

\mainmatter

	\chapter{Pendahuluan}
	Dalam mengkaji struktur bilangan bulat (dan aljabar secara umum), \textit{tools} yang akan sering digunakan adalah sekumpulan objek yang dihimpun menjadi sebuah himpunan dan pemetaan antara himpunan-himpunan tersebut. Adanya pemetaan kemudian memungkinkan kita untuk mendefinisikan operasi pada himpunan. Properti yang dimiliki sebuah himpunan terhadap operasinya inilah yang membentuk struktur yang akan kita kaji.
	\section{Himpunan}
	\subsection{Himpunan dan Konstruksinya}
	Misalkan $S$ adalah koleksi unik dari objek-objek (objek yang sama tidak ditulis dua kali). Kita sebut $S$ adalah suatu himpunan. Untuk suatu objek $s$ di $S$, kita katakan $s$ adalah unsur di $S$ dan kita notasikan $s \in S$.
	\\
	
	Pendefinisian himpunan tidak serta-merta mengakibatkan himpunan tersebut mempunyai isi. Apabila tidak ada objek $s$ (apapun) sehingga $s \in S$, kita katakan $S$ adalah himpunan kosong dan kita notasikan sebagai $\emptyset$. Jika $S$ punya unsur di dalamnya, kita katakan $S$ tak hampa. Beberapa himpunan khusus yang sering muncul kita berikan notasi-notasi khusus yang bisa anda lihat pada bagian notasi di bagian awal catatan ini.
	\\
	
	Untuk bekerja dengan himpunan, cara yang paling dasar dalam mengkonstruksinya adalah dengan melihat unsur di dalamnya. Kemudian, dengan pendefinisian tersebut akan kita buat definisi-definisi untuk membangun himpunan baru.
	\\
	
	Pertama, akan dimulai dengan mendefinisikan subhimpunan. Suatu himpunan $H$ dikatakan subhimpunan dari $S$ jika untuk setiap $h \in H$ berlaku $h \in S$. Dalam kasus dua himpunan $S, S'$ mempunyai tepat anggota sama, dapat kita lihat bahwa $S \subseteq S'$ dan $S' \subseteq S$. Ide inilah yang kita gunakan untuk mendefinisikan kesamaan dua himpunan. Dua himpunan $S, S'$ dikatakan \textbf{sama}, notasikan dengan $S = S'$ jika dan hanya jika $S \subseteq S'$ dan $S' \subseteq S$.
	\\
	
	Terkadang, diperlukan juga untuk mengetahui apakah suatu subhimpunan $H$ bagi $S$ sama dengan $S$. Misalkan $H \subseteq S$ dan $H \ne S$, kita katakan $H$ adalah \textbf{subhimpunan sejati} bagi $S$, notasikan sebagai $H \subset S$.
	\\
	
	Dalam beberapa pembahasan (misalnya terkait komplemen yang akan kita bahas berikutnya), bisa jadi diperlukan suatu ide tentang himpunan semesta $U$ sehingga untuk setiap himpunan yang menjadi bahasan merupakan subhimpunan dari $U$. Misalkan $S$ adalah himpunan bilangan genap, salah satu semesta yang mungkin adalah himpunan $\mathbb{Z}$, yaitu himpunan semua bilangan bulat.
	\\
	
	Untuk suatu himpunan $S$ dan semesta $U$, definisikan komplemen dari $S$, notasikan dengan $S^c$ sebagai himpunan $S^c = \{x \ | \ x \in U, x \notin S \}$
	\\
	
	Dengan konsep-konsep yang kita definisikan tersebut kita dapat konstruksi himpunan - himpunan baru berikut: \begin{itemize}
		\item $A \cup B = \{ x \ | \ x \in A$ atau $x \in B \}$
		\item $A \cap B = \{ x \ | \ x \in A$ dan $x \in B \}$
		\item $A - B = \{ x \ | \ x \in A$ dan $x \notin B \}$
	\end{itemize}
	Dari pendefinisian di atas, kita dapatkan beberapa sifat operasi himpunan yang akan kita himpun dalam teorema berikut
	
	\begin{theorem}[Sifat Operasi Himpunan]
		Misalkan $U$ adalah suatu himpunan semesta dan $X,Y,Z \subseteq U$, sifat-sifat berikut berlaku:
		\begin{enumerate}
			\item $X \cap X^c = \emptyset$
			\item $X \cup X^c = U$
			\item $(X^c)^c = X$
			\item ${(X \cup Y)}^c = X^c \cap Y^c$ dan ${(X \cap Y)}^c = X^c \cup Y^c$
			\item $X \cup (Y \cup Z) = (X \cup Y) \cup Z$ (asosiatif, berlaku pula untuk irisan)
			\item $X \cap Y = Y \cap X$ dan $X \cup Y = Y \cup X$ (komutatif)
			\item $X \cup (Y \cap Z) = (X \cup Y) \cap (X \cup Z)$ (distributif, serupa juga untuk tanda $\cup$ dan $\cap$ yang diganti satu sama lain)
		\end{enumerate}
	\end{theorem}
	\begin{proof}
		Akan kita buktikan dua sifat saja, yaitu sifat pertama dan terakhir. Sisanya diserahkan kepada pembaca.
		\begin{itemize}
			\item Untuk sifat pertama, asumsikan dahulu $X$ tak hampa (jika kosong, trivial). Ambil $x \in X$ (perhatian: jika $X$ kosong kita tidak bisa mengambil $x \in X$, itulah kenapa di awal dinyatakan dulu $X$ tak hampa). Dari definisi $X^c$, jelas bahwa $x$ tidak di $X^c$ sehingga $X \cap X^c = \emptyset$
			\item Untuk sifat terakhir, kita buktikan  $X \cup (Y \cap Z) = (X \cup Y) \cap (X \cup Z)$ dengan membuktikan bahwa himpunan kedua ruas saling \textit{subset}.
			\\
			
			Pertama, akan dibuktikan bahwa  $X \cup (Y \cap Z) \subseteq (X \cup Y) \cap (X \cup Z)$. Ambil $x \in X \cup (Y \cap Z)$. Dari pendefinisian $\cup$, kita tahu bahwa $x \in X$ atau $x \in (Y \cap Z)$. Jika $x \in X$, jelas bahwa $x \in (X \cup Y)$ dan $x \in (X \cup Z)$ sehingga $x \in (X \cup Y) \cap (X \cup Z)$. Untuk $x \in (Y \cap Z)$, berarti $x \in Y$ dan $X \in Z$. Akibatnya, haruslah $x \in (X \cup Y)$ dan $x \in (X \cup Z)$ sehingga $x \in (X \cup Y) \cap (X \cup Z)$. Dengan demikian,  $X \cup (Y \cap Z) \subseteq (X \cup Y) \cap (X \cup Z)$.
			\\
			
			Langkah selanjutnya adalah membuktikan $(X \cup Y) \cap (X \cup Z) \subseteq X \cup (Y \cap Z)$. Ambil $y \in (X \cup Y) \cap (X \cup Z)$. Kita tahu bahwa $y \in (X \cup Y)$ dan $y \in (X \cup Z)$. Jika $y \in X$, jelas bahwa $y \in  X \cup (Y \cap Z)$. Jika tidak, haruslah $y \in Y$ dan $y \in Z$ sehingga $y \in  X \cup (Y \cap Z)$. Jadi, $(X \cup Y) \cap (X \cup Z) \subseteq X \cup (Y \cap Z)$ sehingga dengan menggabungkan kedua hasil kita dapatkan $(X \cup Y) \cap (X \cup Z) = X \cup (Y \cap Z)$. Dengan demikian kita selesai.
		\end{itemize}
	\end{proof}
	Dalam membuktikan suatu sifat bagi operasi himpunan, \textit{guideline} yang umum berlaku adalah dengan memeriksa konstruksi dari himpunannya. Jika anda yakin bukti bisa diberikan dengan memanfaatkan sifat-sifat yang umum dikenal (seperti teorema 1.1.1), silahkan. Namun, jika anda ragu, konstruksilah himpunannya. Sebagai catatan, meskipun diagram venn merupakan alat visualisasi yang baik untuk merepresentasikan himpunan, diagram venn bukanlah alat untuk membuat bukti formal.
	\\
	
	Untuk melatih kemampuan pembuktian anda, cobalah soal berikut
	\begin{enumerate}
		\item Buktikan bagian lain yang belum dibuktikan dari teorema 1.1.1
		\item Misalkan $S$ adalah suatu himpunan dan $A,B \subseteq S$. Definisikan operasi-operasi berikut:
		\begin{itemize}
			\item $A \oplus B = (A-B) \cup (B-A)$
			\item $A \cdot B = A \cap B$
		\end{itemize}
		Tunjukkan bahwa sifat-sifat berikut berlaku:
		\begin{enumerate}
			\item $A \oplus A = \emptyset$
			\item $A \oplus B = A \oplus C$ mengakibatkan $B = C$
			\item $A \cdot (B + C) = (A \cdot B) + (A \cdot C)$
		\end{enumerate}
	\end{enumerate}
	Cara lain untuk mengonstruksi himpunan adalah dengan hasil kali kartesian. Misalkan $A,B$ suatu himpunan yang keduanya tak hampa. Hasil kali kartesian $A,B$, notasikan sebagai $A \times B$ adalah
	$$A \times B = \{ (a,b) \ | \ a \in A,\ b \in B \}$$
	Salah satu penggunaan hasil kali kartesian adalah dalam pendefinisian relasi. Misalkan $S$ adalah suatu himpunan. Untuk $R \subset S \times S$ tak hampa, kita sebut $R$ sebagai suatu relasi. Kita katakan $a,b \in S$ berelasi dengan relasi $R$ (notasikan sebagai $a \ R \ b$) jika $(a,b) \in R$.
	\\
	
	Jika sebelumnya kita bicara tentang konstruksi himpunan, sekarang kita akan meninjau objek di dalamnya. Definisikan \textbf{kardinalitas} dari suatu himpunan $S$, notasikan sebagai $|S|$ sebagai banyaknya objek di dalam himpunan $S$. Untuk $S$ yang berhingga, tentu mudah untuk membandingkan banyak objeknya. Namun, bagaimana dengan yang tak berhingga? Kita akan bahas kemudian.
	\subsection{Urutan}
	Bahasan pada bagian ini akan kita batasi pada urutan total (\textit{total order}). Misalkan $S$ sebuah himpunan. Definisikan $\preceq$ sebuah relasi \textit{total order} di $S$, untuk setiap $a,b,c \in S$ berlaku
	\begin{itemize}
		\item Jika $a \preceq b$ dan $b \preceq a$ maka $a = b$
		\item Jika $a \preceq b$ dan $b \preceq c$ maka $a \preceq c$
		\item $a \preceq b$ atau $b \preceq a$
	\end{itemize}
	Jenis lain dari urutan adalah urutan parsial (\textit{partial ordering}) yang merupakan perumuman dari urutan total. Dalam kasus urutan parsial, tidak harus semua unsur di $S$ dapat dibandingkan (tetapi jika bisa akan seperti urutan total). Urutan seperti ini dapat digunakan untuk mendefinisikan urutan bagi himpunan. Misal dua buah himpunan $A,B$, kita katakan $A \preceq B$ jika $A \subseteq B$. Andaikan $A,B$ disjoint, kita tidak dapat mengurutkan $A$ dan $B$.
	\\
	
	Berikutnya kita akan tinjau properti yang dimiliki oleh bilangan asli.
	
	\begin{inneraxiom}[Well-Order]
		Setiap $S$ subset $\mathbb{N}$ yang tak hampa punya unsur terkecil
	\end{inneraxiom}
	Pernyataan yang setara (dan lebih umum) dengan prinsip \textit{well order} ini adalah sebuah himpunan $S$ dengan urutan total $\preceq$ dikatakan \textit{well-ordered} (terurut dengan baik) jika ada unsur $x \in S$ sehingga untuk setiap $s \in S$ berlaku $x \preceq s$ (yakni $S$ punya unsur terkecil).
	\\
	
	Prinsip \textit{well order} ini dapat digunakan untuk membuktikan teorema-teorema dalam teori bilangan atau hal yang berkaitan dengan bilangan bulat, termasuk prinsip induksi. Sebagai contoh, kita akan membuktikan bahwa tidak ada bilangan bulat di antara $0$ dan $1$.
	\\
	
	Definisikan $S$ sebagai himpunan bilangan bulat antara $0$ dan $1$. Andaikan $S$ tak hampa. Dari prinsip \textit{well order}, ada unsur terkecil, katakan saja $s$, di $S$. Jelas berlaku $0 < s < 1$ sehingga $0 < s^2 < s$. Padahal $s$ unsur terkecil di $S$. Kontradiksi. Akibatnya, $S$ haruslah kosong. Jadi, tidak ada bilangan bulat di antara $0$ dan $1$.
	\section{Pemetaan dan Operasi}
	\subsection{Pemetaan}
	Misalkan $S,T$ suatu himpunan tak hampa. Kita tentunya dapat mengaitkan anggota himpunan $S$ dengan anggota himpunan $T$, notasikan sebagai $f: S \rightarrow T$. Pengaitan seperti ini secara umum dapat bebas didefinisikan. Namun, kita akan membahas pengaitan khusus yang kita sebut pemetaan. Sebuah pengaitan $f: S \rightarrow T$ adalah suatu pemetaan jika dan hanya jika untuk setiap unsur $s \in S$ dipetakan ke satu unsur di $T$. Unsur hasil pemetaan $s$ oleh $f$ kita notasikan sebagai $f(s)$. Tentunya $f(s) \in T$. Akan berguna juga untuk mendefinisikan himpunan Peta($f$), yaitu himpunan
	$$\textnormal{Peta}(f) = \{ f(s) \, | \, s \in S \}$$
	Karena $f(s) \in T$, jelas bahwa Peta$(f) \subseteq T$. Beberapa referensi lain (terutama yang berbahasa Inggris) menotasikannya juga sebagai Im($f$) atau Image($f$).
	\\
	
	Dua buah pemetaan $f,g: S \rightarrow T$ kita katakan sama jika ia memetakan semua unsur di $S$ ke unsur yang sama di $T$. Dengan kata lain, $f = g$ jika $f(s) = g(s)$ untuk setiap $s \in S$.
	\\
	
	Dua buah pemetaan $f: S \rightarrow T$ dan $g: U \rightarrow V$ dapat juga dikomposisikan, asalkan Peta($f$) $ \cap \ U \ne \emptyset$. Untuk komposisi $g \circ f: S \rightarrow V$, Peta($g \circ f$) adalah $\{ g(x) : x \in \textnormal{Peta}(f) \cap U \}$. Komposisi tersebut juga dapat ditulis sebagai pemetaan baru, notasikan sebagai $gf$ sehingga diagram berikut komutatif (untuk diagram ini, asumsikan $T = $ Peta($f$) $ = U$ untuk mempermudah):\\
	\adjustbox
	{scale=1.4,center}{%
		\begin{tikzcd}
			S \arrow[rd,swap,"gf"]
			\arrow{r}{f} &
			T \arrow{d}{g} \\
			& V
		\end{tikzcd}
	}
	\\
	
	(Diagram komutatif di sini berarti untuk sembarang titik awal dan akhir, setiap rutenya merepresentasikan hal yang sama --- hal ini akan berguna untuk diagram yang lebih rumit)
	\\
	
	Berikutnya kita akan meninjau beberapa pemetaan khusus dari perilakunya.
	\\
	
	\textbf{Definisi. }Sebuah pemetaan $f: S \rightarrow T$ dikatakan \textbf{satu-satu} atau \textbf{injektif} jika untuk $s \ne s' \in S$ berlaku $f(s) \ne f(s')$.
	\\
	
	Pernyataan yang ekuivalen untuk pernyataan tersebut adalah untuk $f$ satu-satu, $f(s) = f(s')$ mengimplikasikan $s = s'$ untuk $s,s' \in S$. Untuk $f$ pemetaan satu-satu, kita dapatkan teorema berikut
	\begin{theorem}
		Misal $f: S \rightarrow T$ satu-satu, maka $|S| \le |T|$
	\end{theorem}
	\begin{proof}
		Misalkan $|S| > |T|$, dari prinsip sarang merpati, akan ada $t \in T$ sehingga ada $x,y \in S$ dengan $x \ne y$ dan $f(x) = f(y) = t$. Akibatnya, $f$ tidak satu-satu. Teorema 1.2.1 adalah kontraposisi pernyataan ini.
	\end{proof}
	Teorema ini beserta analoginya yang akan kita bahas berikutnya akan membantu kita untuk memeriksa kardinalitas suatu himpunan.
	\\
		
	\textbf{Definisi. }Sebuah pemetaan $f: S \rightarrow T$ dikatakan \textbf{pada} atau \textbf{surjektif} jika untuk setiap $t \in T$ ada $s \in S$ sehingga $f(s) = t$.
	\\
	
	Analog dengan teorema 1.2.1, kita punya teorema berikut
	\begin{theorem}
		Misal $f: S \rightarrow T$ pada, maka $|T| \le |S|$
	\end{theorem}
	\begin{proof}
		Latihan. (Jalan buktinya serupa dengan teorema 1.2.1)
	\end{proof}
	\textbf{Definisi. }Sebuah pemetaan $f: S \rightarrow T$ dikatakan \textbf{satu-satu dan pada} atau \textbf{bijektif} jika $f$ satu-satu dan pada
	\\
	
	Menggabungkan teorema 1.2.1 dan teorema 1.2.2, kita dapatkan akibat berikut
	\begin{corollary}
		Misal $f: S \rightarrow T$ suatu bijeksi, maka $|S| = |T|$
	\end{corollary}
	Dari akibat ini, untuk memeriksa apakah kardinalitas dari dua himpunan sama, kita cukup mencari suatu bijeksi dari satu himpunan ke himpunan lain. 
	\\
	
	\textbf{Diskusi}
	\begin{enumerate}
		\item Carilah sebuah bijeksi dari $\mathbb{N}$ ke $\mathbb{Z}$
		\item Apakah ada bijeksi dari $\mathbb{N}$ ke $\mathbb{N} \times \mathbb{N}$? Jika ya, beri contoh bijeksinya.
		\item Apakah ada bijeksi dari $\mathbb{N}$ ke $\mathbb{Q}$? Jika ya, beri contoh bijeksinya.
		\item Misalkan $f,g: S \rightarrow S$ suatu pemetaan dengan $g \circ f$ konstan. Untuk $g$ pemetaan pada, apa yang dapat anda simpulkan tentang $f$? Bagaimana untuk $f$ satu-satu?
	\end{enumerate}

	Sebuah pemetaan juga dapat dikenakan pada suatu himpunan. Misalkan $f: S \rightarrow T$ dan $A \subseteq S$. Hasil pemetaan $f(A)$ adalah $$f(A) = \{ f(a) : a \in A \}$$
	Misal untuk pemetaan dan himpunan yang sama, $B \subseteq T$. Tinjau himpunan $$H = \{ s \in S \, | \, f(s) \in B \}$$
	Himpunan $H$ dapat kita notasikan sebagai $H = f^{-1}(B)$ dengan $f^{-1}$ adalah suatu pemetaan $f^{-1}: T \rightarrow S$.
	\\
	
	Pemetaan $f^{-1}: T \rightarrow S$ disebut sebagai \textbf{invers} dari pemetaan $f: S \rightarrow T$ jika
	$$f^{-1} \circ f = i_S$$ dan $$f \circ f^{-1} = i_T$$
	dengan $i_S$ dan $i_T$ masing-masing adalah pemetaan identitas di $S$ dan $T$ (yaitu pemetaan yang memetakan setiap unsur di suatu himpunan dengan dirinya sendiri). Perhatikan bahwa agar definisi ini terpenuhi, $f$ haruslah satu-satu dan pada (silahkan periksa). Untuk kasus khusus ketika $f$ adalah pemetaan dari suatu himpunan ke dirinya sendiri dan $f = f^{-1}$, kita sebut $f$ sebagai \textbf{involusi}.
	\\
	
	\textbf{Diskusi. }Jika $f$ satu-satu dan pada, apakah $f^{-1}$ juga harus satu-satu dan pada? Buktikan.
	\subsection{Operasi}
	Misalkan $S$ adalah suatu himpunan tak hampa. Secara umum, suatu operasi di $S$ adalah sebuah pemetaan $S^k \rightarrow S$ untuk suatu $k$ bulat nonnegatif (pangkat di sini adalah hasil kali kartesian berulang). Pada bagian ini, pembahasan akan difokuskan pada \textbf{operasi biner}, yaitu untuk $k = 2$.
	\\
	
	Definisikan pemetaan
	\begin{equation*}
		\begin{split}
			& \cdot: S \times S \rightarrow S\\
			& (a,b) \mapsto a \ \cdot \ b
		\end{split}
	\end{equation*}
	Yaitu pemetaan yang mengirimkan $(a,b) \in S \times S$ ke $a \cdot b \in S$. Kita katakan $\cdot$ adalah suatu operasi jika untuk setiap $(a,b) \in S \times S$ mengakibatkan $a\cdot  b \in S$ dan terdefinisi dengan baik. Dengan kata lain, suatu operasi haruslah tertutup atas himpunannya. Suatu operasi terdefinisi dengan baik berarti untuk $a = a' \in S$ dan $b = b' \in S$, $(a,b)$ haruslah dipetakan ke hal yang sama dengan hasil pemetaan $(a',b')$. Misalnya, $a$ dan $a'$ adalah wakil kelas ekuivalen berbeda dari suatu kelas ekuivalen, yaitu 2 hal berbeda yang merepresentasikan hal yang sama (akan dibahas kemudian di bab 3). Jadi, suatu operasi haruslah tidak bergantung pada representasi.
	\\
	
	Suatu himpunan tak hampa $H$ yang dilengkapi dengan operasi biner $\oplus$ disebut sebagai sebuah \textbf{sistem matematika}. Kita notasikan sistem matematika tersebut sebagai $(H, \oplus)$. Sebagai contoh, himpunan bilangan bulat $\mathbb{Z}$ yang dilengkapi operasi penjumlahan  dan perkalian (biasa) + dan $\times$ membentuk sistem matematika $(\mathbb{Z}, +)$ dan $(\mathbb{Z}, \times)$. Apabila operasi yang dimaksud jelas, penulisan sistem matematika $(H, \oplus)$ dapat diringkas menjadi $H$.
	\\
	
	Sebagai perumuman, suatu sistem matematika dapat didefinisikan dengan lebih dari 1 operasi biner atas himpunannya. Sebagai contoh adalah sistem matematika $(\mathbb{Z},+,\times)$, yaitu himpunan bilangan bulat yang dilengkapi operasi penjumlahan dan perkalian (untuk saat ini kita belum bicara apapun tentang strukturnya.)
\chapter{Sistem Bilangan Bulat}
	Bahasan struktur bilangan bulat akan dimulai dari hal yang paling familiar dengan kita, yaitu sistem bilangan bulat beserta sifat-sifatnya.
	\section{Bilangan Bulat}
	Tinjau sistem matematika $(\mathbb{Z},+,\times)$ dengan $+$ dan $\times$ adalah operasi penjumlahan dan perkalian biasa. Kita punya sifat-sifat berikut untuk setiap $a,b,c \in \mathbb{Z}$:
	\begin{enumerate}
		\item Sifat asosiatif, yakni $(a+b)+c = a+(b+c)$ dan $a \times (b \times c) = (a \times b) \times c$
		\item Sifat komutatif, yakni $a+b = b+a$ dan $a \times b = b \times a$
		\item Ada unsur identitas $e \in \mathbb{Z}$ sehingga $ae = ea = a$. Untuk penjumlahan $e = 0$ sehingga $a + 0 = 0 + a = 0$ dan untuk perkalian $e = 1$ sehingga $a \times 1 = 1 \times a = a$
		\item Ada unsur invers penjumlahan $-a$ sehingga $a + (-a) = (-a) + a = 0$
		\item Sifat distributif, yakni $a \times (b + c) = a\times b + a \times c$
	\end{enumerate}
	Untuk setiap sistem matematika yang memenuhi sifat-sifat tersebut, kita sebut sebagai \textbf{gelanggang} (\textit{ring}). Perhatikan pula bahwa pada gelanggang bilangan bulat, misalkan $a,b \in \mathbb{Z}$ dengan $ab = 0$. Haruslah $a = 0$ atau $b = 0$. Sifat ini tidak selalu berlaku pada sembarang gelanggang (misalnya pada perkalian matriks). Gelanggang yang memenuhi sifat tersebut kita sebut sebagai \textbf{daerah integral} (\textit{integral domain}).
	\\
	
	Berikutnya, kita akan buktikan hasil yang selama ini sudah kita kenal (tetapi sebenarnya bukanlah aksioma)
	
	\begin{theorem}
		Untuk $a,-a,b \in \mathbb{Z}$ berlaku
		\begin{itemize}
			\item $0a =0$
			\item $(-a)b = a(-b) = -(ab)$
		\end{itemize}
	\end{theorem}
	\begin{proof}
		\begin{enumerate}
			\item Dari definisi, $0 + 0 = 0$. Kita dapatkan
			\begin{equation*}
				\begin{split}
				0a & = (0+0)a\\
				& = 0a + 0a\\
				\end{split}
			\end{equation*}
			Menjumlahkan kedua ruas dengan $-(0a)$, kita dapatkan $0a = 0$.
		\item Dari sifat di atas, kita tahu bahwa $a + (-a) = 0$ dan $0b = 0$ sehingga $0 = (a + (-a))b = ab + (-a)b$ (sifat distributif). Dengan menjumlahkan kedua ruas dengan $-ab$ kita dapatkan
		\begin{equation*}
		\begin{split}
			-ab & = ab + a(-b) + (-ab)\\
			-ab & = a(-b) 
		\end{split}
		\end{equation*}
		Dengan cara serupa, $0 = (a + (-a))b = ab + a(-b)$ (perkalian bersifat komutatif) sehingga
		\begin{equation*}
			\begin{split}
				0 & = ab + a(-b)\\
				-ab & = -ab + ab + a(-b)\\
				-ab & = a(-b)
			\end{split}
		\end{equation*}
		Dengan demikian, $(-a)b = a(-b) = -ab$
		\end{enumerate}
	
	Jadi, teorema tersebut terbukti.
	\end{proof}
	
	Berikutnya, kita akan definisikan keterbagian di bilangan bulat. Misal $a,b \in \mathbb{Z}$, definisikan
	\begin{enumerate}
		\item Unsur $a$ dikatakan membagi $b$, notasikan sebagai $a | b$, jika ada $c \in \mathbb{Z}$ sehingga $b = ac$
		\item Unsur $a$ dikatakan unit jika ada $b$ bulat sehingga $ab = 1$. Dapat diperiksa bahwa unit di $\mathbb{Z}$ hanyalah -1 dan 1
		\item Nilai mutlak dari $a$ adalah $$ |a| =  
		\begin{cases}
			a &\mbox{untuk $a \ge 0$}\\
			-a &\mbox{untuk $a < 0$}
		\end{cases}	$$
	\end{enumerate}
	
	Kemudian kita akan bahas tentang algoritma pembagian. Hal ini akan sering digunakan ketika kita membahas pembagi sekutu terbesar.
	\begin{theorem}[Algoritma Pembagian]
		Misal $a,b \in \mathbb{Z}$ dengan $a \ne 0$. Ada $p,r \in \mathbb{Z}$ unik sehingga $$b = ap + r$$ dengan $0 \le r < |a|$
	\end{theorem}
	\begin{proof}
		Kita perlu tunjukkan dulu $p,r$ ada, lalu tunjukkan bahwa $p,r$ unik. Tinjau dahulu kasus $a > 0$. 
		Definisikan himpunan $S = \{ b - ka \ | \ b-ka \ge 0, k \in \mathbb{Z} \}$. Dari \textit{well-order} kita tahu bahwa $S$ punya unsur terkecil asal $S$ tak hampa. Ini akan kita buktikan dulu. Untuk $b \ge 0$ jelas $S$ tak hampa (tinjau $k = 0$). Untuk $b < 0$ tinjau $k = b$ sehingga $b - ka = b - ba = b(1-a)$. Karena $a > 0$ dan $a$ bulat pastilah $(1-a)$ nol atau negatif sehingga $b(1-a) \ge 0$. Jadi, $S$ tidak kosong. Dengan demikian, kita bisa menggunakan prinsip \textit{well-order} untuk mendapatkan $S$ punya unsur terkecil. Misalkan unsur terkecilnya adalah $$r = b - k'a$$ tentu $0 \le r < a$. Karena $r \in S$, $r$ tentu non-negatif. Cukup dibuktikan $r < a$. Misal sebaliknya, $r \ge a$, berarti $b - k'a \ge a \iff b - (k'+1)a \ge 0 \in S$. Berarti, $r$ bukan unsur terkecil sehingga terjadi kontradiksi dengan prinsip \textit{well-order}. Jadi, haruslah $r < a$. Untuk mencari $p$, mudah dilihat bahwa $p = k'$ sehingga ada $p,r$ yang membuat $b = ap + r$. (Untuk $a < 0$ serupa)
		\\
		
		Kemudian untuk menunjukkan $p,r$ unik, perlu ditunjukkan untuk $b = ap_1+r_1$ dan $b = ap_2 + r_2$ berlaku $p_1 = p_2$ dan $r_2 = r_1$. Tanpa mengurangi keumuman, asumsikan $r_1 \ge r_2$. Kita dapatkan $ap_1 + r_1 = ap_2 + r_2$ sehingga $r_1 - r_2 = a(p_2 - p_1)$. Jadi, $r_1 - r_2$ kelipatan $a$. Karena $0 \le r_1, r_2 < a$ dan $r_2 \ge r_1$, haruslah $0 \le r_2 - r_1 < a$. Satu-satunya kelipatan $a$ pada \textit{range} tersebut adalah $0$ sehingga $r_1 - r_2 = 0 \iff r_1 = r_2$. Akibatnya juga haruslah $p_1 = p_2$. Jadi, $p,r$ unik.
	\end{proof}
	
	Berikutnya, kita akan buktikan eksistensi generator atau pembangun bagi subhimpunan $\mathbb{Z}$.
	\begin{theorem}
		Untuk $S \subseteq \mathbb{Z}$ tak hampa dengan $a + b, a - b$ di $S$ untuk setiap $a,b$ di $S$, ada $n \in S$ sehingga $$S = \{kn \, | \, k \in \mathbb{Z}\}$$
	\end{theorem}
	\begin{proof}
		Ambil $a \in S$, jelas $a-a = 0 \in S$. Untuk $S = {0}$, kita bisa pilih $n = 0$ sehingga $S = \{k0 \, | \, k \in \mathbb{Z} \}$. Untuk kasus lainnya, misal $a \ne 0$, berarti $-a = 0 - a \in S$ sehingga $\{-a, 0, a\} \subseteq S$. Konsekuensinya, himpunan $S$ memuat bilangan positif. Definisikan himpunan $T = \{ t \, | \, t \in S, t > 0 \}$. Misalkan $x$ unsur terkecil di $T$. Untuk $k = 0$, $kx = 0$. Untuk $k > 0$, $kx = x + ... + x$ (sebanyak $k$ kali). Untuk $k < 0$, ada $m$ bulat positif sehingga $k = -m$ dan $kx = (-m)x = m(-x) = (-x) + ... + (-x)$ (sebanyak $m = -k$ kali). Akibatnya, $\{ kx \, | \, k \in \mathbb{Z} \} \subseteq S$. Ambil $y \in S$, ada $p,r \in \mathbb{Z}$ sehingga $y = px + r$ dengan $0 \le r \le x$. Jelas bahwa $r = y - px \in S$. Karena $x$ unsur terkecil di $S$ dan $0 \le r \le x$, haruslah $r = 0$. Dengan demikian, $y = px \in S$ sehingga $S = \{kn \, | \, k \in \mathbb{Z}\} $
	\end{proof}
	Definisi-definisi berikut akan membantu untuk bahasan subbab berikutnya.
	\\
	\begin{enumerate}
		\item Unsur $a,b \in \mathbb{Z}$ dikatakan \textbf{sekawan} jika $a|b$ dan $b|a$
		\item Unsur $p \in \mathbb{Z}$ bukan unit dikatakan prima jika $p | ab$ mengimplikasikan $p | a$ atau $p | b$
	\end{enumerate}
	\section{Pembagi Sekutu Terbesar}
	Tinjau dulu pembagi sekutu terbesar dari 2 bilangan bulat. Untuk $a,b,d \in \mathbb{Z}$. Unsur $d$ dikatakan pembagi sekutu terbesar bagi $a,b$ jika $d | a$ dan $d|b$ dan jika ada $c$ bulat yang menyebabkan $c | a$ dan $c | b$ maka $c | d$. Kita notasikan sebagai $(a,b) = d$. Beberapa referensi ada yang menuliskannya sebagai $GCD(a,b)$ (GCD di sini adalah \textit{Greatest Common Divisor}) ataupun $FPB(a,b)$. Notasi $()$ dipilih di sini untuk memudahkan saja.\\
	
	Kita definisikan juga untuk dua buah $a,b$ jika $(a,b) = 1$ kita katakan $a,b$ \textbf{relatif prima} atau \textbf{saling prima}.
	\\
	
	Perlu diperhatikan bahwa mendefinisikan GCD tidak serta-merta mengimplikasikan GCD ada. Kali ini kita akan buktikan eksistensinya.
	
	\begin{theorem}[Eksistensi GCD]
		Untuk $a,b \in \mathbb{Z}$ tidak keduanya nol ada $d \in \mathbb{Z}$ unik sehingga $d = (a,b)$ dan $d$ dapat ditulis sebagai $d = m_0a + n_0b$ untuk suatu $m_0,n_0 \in \mathbb{Z}$
	\end{theorem}
	
	\begin{proof}
		 Definisikan $A = \{ma + nb \ | \ m,n \in \mathbb{Z}\}$. karena $a,b$ tak keduanya nol, $A$ punya unsur tak nol. Jika $x \in A$ dengan $x < 0$, maka $-x = (-m_1)a + (-n_1)b \in A$ untuk suatu $m_1, n_1$ bulat. Jadi, $A$ punya unsur positif. Dari \textit{well-order}, $A$ punya unsur positif terkecil, sebut saja $c$. Karena $c \in A$, ada $m_0, n_0$ sehingga $m_0a + n_0b = c$. Klaim: $c$ adalah GCD dari $a,b$. Perhatikan bahwa jika $d | a$ dan $d |b$, haruslah $d | (m_0 a + n_0b)$ sehingga $d|c$. Dengan algoritma pembagian, dapat ditunjukkan bahwa $c | a$ dan $c | b$. Karena $c$ unsur positif terkecil di $A$, dan $c | a$ dan $c | b$, haruslah $c = (a,b)$, yaitu GCD dari $a,b$. Jadi, GCD ada.
		 \\
		 
		 Untuk membuktikan GCD unik, misal ada $t > 0$ sehingga $t$ juga memenuhi definisi GCD (selain $c$). Haruslah $t | c$ dan $c | t$, sehingga $c = \pm t$. Namun, $c$ dan $t$ keduanya positif, sehingga $c = t$. Jadi, GCD unik.
	\end{proof}
	
	Kemudian tinjau untuk lebih dari $2$ unsur. Misalkan $\{ a_1, a_2, ..., a_n \} \subseteq \mathbb{Z}$ dengan $a_i \ne 0$. Definisikan unsur $d \in \mathbb{Z}$ sebagai pembagi sekutu terbesar bagi $a_1, a_2, ..., a_n$ jika
	\begin{itemize}
		\item $d | a_i$ untuk setiap $i \in {1,2,...,n}$
		\item Jika ada $c | a_i$ untuk setiap $i \in {1,2,...,n}$, maka $c | d$
	\end{itemize}
	Kita notasikan $(a_1, a_2, ..., a_n) = d$.
	\\
	
	\textbf{Diskusi. }Coba anda buktikan bahwa GCD untuk $n$ unsur ada.
	\section{Bilangan Prima dan Faktorisasi}
	Kita awali dahulu bahasan pada subbab ini dengan membahas bilangan prima. Suatu bilangan $p$ (untuk sekarang asumsikan $p \in \mathbb{Z}$) dikatakan \textbf{prima} jika $p | ab$ mengakibatkan $p | a$ atau $p | b$ dan $p$ bukan unit (sehingga di bilangan bulat, $1$ dan $-1$ bukanlah prima).\\
	
	Di sekolah dasar dan menengah, kita sudah cukup familiar dengan konsep faktorisasi prima dari suatu bilangan. Namun, dari mana kita yakin bahwa setiap bilangan bulat dapat difaktorkan menjadi faktor-faktor prima (anggap satu dan nol sebagai perkalian kosong)? Kita tinjau dahulu kasus untuk bilangan bulat positif lebih dari 1. Untuk itu, kita punyai teorema berikut:
	\begin{theorem}[Teorema Dasar Aritmatika]
		Setiap bilangan bulat lebih dari 1 adalah bilangan prima atau merupakan hasil perkalian bilangan prima secara tunggal
	\end{theorem}
	Untuk membuktikan teorema ini, ada dua hal yang perlu diperhatikan. Pertama, menunjukkan bahwa faktorisasinya ada, kemudian menunjukkan bahwa faktorisasi tersebut tunggal. Dengan kata lain, apabila suatu bilangan $x \in \mathbb{Z}$ punya lebih dari satu cara untuk difaktorkan sedemikian sehingga $x = p_1 p_2 ... p_n = q_1 q_2 ... q_m$, haruslah berlaku $m = n$ dan ada suatu permutasi $\sigma : [1,n] \rightarrow [1,n]$ sehingga $p_i = q_{\sigma{(i)}}$ (perbedaan hanya pada urutan perkalian saja).
	\begin{proof}
		Kita mulai buktinya dengan membuktikan eksistensi faktor prima. Tinjau $n = 2$, karena 2 prima, jelas 2 dapat ditulis sebagai perkalian faktor-faktor prima (yakni dirinya sendiri). Andaikan untuk setiap $n < k$ benar bahwa setiap bilangan bulat positif lebih dari 1 punya faktor pirma. Untuk $k$ prima, jelas. Untuk $k$ tidak prima, haruslah ada $x,y$ dengan $1 < x \le y < k$ dengan $k = xy$. Dari asumsi kita sebelumnya, $x$ dan $y$ dapat ditulis sebagai perkalian faktor-faktor prima. Akibatnya, $k$ juga dapat ditulis sebagai perkalian faktor-faktor prima. Dari prinsip induksi kuat, kita simpulkan setiap bilangan bulat lebih dari 1 dapat ditulis sebagai perkalian faktor-faktor prima.\\
		
		Berikutnya, kita akan tinjau ketunggalannya. Andaikan ada suatu bilangan yang punya lebih dari 1 cara untuk memfaktorkannya, dari prinsip \textit{well-ordering} haruslah ada bilangan terkecil yang demikian, sebut saja $x$. Misalkan $x = p_1 p_2 ... p_n = q_1 q_2 ... q_m$. Akibatnya, $p_1 | q_1 q_2 ... q_m$. Karena $p_1$ prima, ada $i \in [1,m]$ sehingga $p_1 | q_i$. Tanpa mengurangi keumuman, misalkan $i = 1$ sehingga $p_1 | q_1$. Karena keduanya prima, haruslah $p_1 = q_1$. Kita dapatkan $p_2 p_3 ... p_n = q_2 q_3 ... q_m < x$. Kontradiksi dengan fakta bahwa $x$ adalah bilangan terkecil yang punya lebih dari satu cara untuk difaktorisasi prima. Kita simpulkan bahwa faktorisasi prima tersebut haruslah tunggal.
	\end{proof}
	Ternyata faktorisasi tunggal tidak selalu berlaku di setiap sistem matematika. Sebagai contoh, jika kita meninjau sistem matematika yang dibangun oleh himpunan $\{ a+b\sqrt{-5} \, | \, a,b \in \mathbb{Z} \}$, kita dapat lakukan faktorisasi $6 = 2 \times 3 = (1+\sqrt{-5})(1-\sqrt{-5})$. Lebih lanjut tentang ini akan kita bahas ketika kita mengkaji daerah faktorisasi tunggal di bab 4.\\
	
	Berikutnya, kita akan tinjau akibat dari teorema sebelumnya. Hasil ini dibahas pertama kali pada buku \textit{Elements} karya Euclid.
	\begin{theorem}
		Ada tak hingga banyaknya bilangan prima
	\end{theorem}
	\begin{proof}
		Andaikan ada berhingga bilangan prima, definisikan himpunan $P = \{ p_1, p_2, ..., p_n \}$ sebagai himpunan semua bilangan prima. Berikutnya, tinjau $k = p_1 p_2 ... p_n + 1$. Jelas $k > 1$. Dari teorema dasar aritmatika, $k$ punya faktor prima, sebut saja $p$. Andaikan ada $i \in [1,n]$ sehingga $p = p_i$, haruslah $p|1$. Akibatnya, haruslah $p \ne p_i$ untuk setiap $i \in [1,n]$. Jadi, ada $p$ prima dengan $p \not\in P$, himpunan semua bilangan prima. Kontradiksi. Akibatnya, haruslah ada tak hingga banyaknya bilangan prima.
	\end{proof}
	
\chapter{Sistem Bilangan Bulat Modulo}
	Setelah sebelumnya mengkaji sistem bilangan bulat yang biasa kita kenal, kali ini kita akan membahas tentang sistem bilangan bulat modulo.
	\section{Relasi Ekuivalen dan Modulo}
	\subsection{Relasi Ekuivalen}
	Kita awali pembahasan di bagian ini dengan membahas kembali tentang relasi. Misalkan $A,B$ adalah himpunan tak hampa. Hasil kali kartesian kedua himpunan tersebut adalah
	$$A \times B = \{(a,b) \, | \, a \in A \ , b \in B\}$$
	Kita dapat definisikan subset tak hampa dari $A\times B$, sebut $R \subseteq A\times B$ sebagai suatu \textbf{relasi}. Umumnya, relasi akan kita notasikan sebagai $a \ R \ b$, yang berarti $(a,b) \in R$
	Misalkan $\sim$ adalah sebuah relasi terhadap anggota himpunan $S$, yaitu $\sim \ \subseteq S \times S$. Relasi $\sim$ dikatakan relasi ekuivalen jika:
	\begin{itemize}
		\item $(x,x) \in \ \sim$ untuk setiap $x \in S$ (yakni, $x \sim x$)
		\item $(x,y) \in \ \sim \iff (y,x) \in \ \sim$ (yakni, $x \sim y \iff y \sim x$)
		\item $(x,y) \in \ \sim$ dan $(y,z) \in \ \sim \implies (x,z) \in \ \sim$ (yakni, $x \sim y$ dan $y \sim z \implies x \sim z$) 
	\end{itemize}
	Sifat pertama dikenal sebagai sifat \textbf{refleksif}, sifat kedua dikenal sebagai sifat \textbf{simetris}, dan sifat ketiga dikenal sebagai sifat \textbf{transitif}.
	\\

	Sekarang, relasi seperti apakah yang memenuhi sifat tersebut? Kita ambil contoh yang sederhana: relasi "=" pada himpunan bilangan bulat. Mudah diperiksa bahwa relasi tersebut adalah relasi ekuivalen. Contoh lainnya adalah definisikan relasi $\sim$ pada himpunan matriks $n \times n$ simetrik dengan koefisien bilangan bulat $\mathcal{M} \subseteq M_{n\times n}(\mathbb{Z})$ dengan $A \sim B \iff A = B^{T}$ (matriks simetris adalah matriks yang sama dengan transposenya). Akan diperiksa bahwa $\sim$ adalah relasi ekuivalen.
	\\
	
	Untuk memeriksa suatu relasi adalah relasi ekuivalen, kita bisa periksa ketiga sifat relasi ekuivalen. Tinjau relasi $\sim$ yang didefinisikan sebelumnya. 
	\\
	
	Pertama, akan ditunjukkan bahwa relasi tersebut refleksif. Ambil $A \in \mathcal{M}$. Karena $A$ simetrik, jelas bahwa $A = A^{T}$. Akibatnya, untuk sembarang $A \in \mathcal{M}$ berlaku $A \sim A$.
	\\
	
	Berikutnya, akan ditunjukkan bahwa relasi tersebut simetris. Ambil $A,B \in \mathcal{M}$. Jika $A \sim B$, berarti $A = B^{T}$. Karena $A,B$ simetrik, kita punya $A^T = A$ dan $B^T = B$ sehingga $B = A^T$. Jadi, $A \sim B \iff B \sim A$
	\\
	
	Terakhir, akan ditunjukkan bahwa relasi tersebut bersifat transitif. Ambil $A,B,C \in \mathcal{M}$. Jika $A \sim B$ dan $B \sim C$, berarti $A = B^T$ dan $B = C^T$. Akibatnya, $A^T = B = C^T$. Karena $A$ simetrik, $A = A^T$ sehingga $A = C^T$. Dengan demikian, $A \sim C$.
	\\
	
	Karena ketiga properti relasi ekuivalen dipenuhi, relasi $\sim$ tersebut merupakan relasi ekuivalen.
	\\
	
	Untuk membuktikan bahwa suatu relasi bukanlah relasi ekuivalen, anda cukup mencari penyangkal bagi salah satu properti relasi ekuivalen. Sebagai gambaran, silahkan anda periksa apakah relasi-relasi berikut merupakan relasi ekuivalen (semua variabel dan relasi di bilangan bulat, kecuali disebutkan sebaliknya):
	\begin{enumerate}
		\item Relasi $\sim$ dengan $x \sim y \iff x - y$ ganjil
		\item Relasi $\sim$ dengan $x \sim y \iff x - y$ genap
		\item Untuk $x,y \in \mathbb{R}$, relasi $\sim$ di $\mathbb{R}$ dengan $x \sim y \iff x -y$ rasional
		\item Untuk $x,y \in \mathbb{R}$, relasi $\sim$ di $\mathbb{R}$ dengan $x \sim y \iff x < y$ dan $y < x$
		\item Misalkan $S \subseteq \mathbb{Z}$ tak hampa dengan penjumlahan dan pengurangan tertutup di $S$ (yakni, untuk $a,b \in S, \ a+b \in S$ dan $a-b \in S$). Relasi $\sim$ dengan $x \sim y \iff x-y \in S$  (Relasi seperti ini akan anda temui lagi saat membahas koset dari suatu grup)
	\end{enumerate}
	\subsection{Kelas Ekuivalen}
	Misalkan $\sim$ adalah suatu relasi ekuivalen di himpunan $S$. Definisikan kelas ekuivalen untuk suatu $s \in S$ sebagai 
	$$\kappa(s) = \{ x \in S \ | \ x \sim s \}$$
	Yaitu himpunan semua anggota $S$ yang ekuivalen dengan $s$. Tinjau teorema berikut.
	
	\begin{theorem}
		\label{kardKlasEq}
		Misalkan $\sim$ adalah suatu relasi ekuivalen pada $S$, untuk setiap $x,y \in S$ berlaku tepat satu:
		$\kappa(x) \cap \kappa(y) = \emptyset$ atau
		$\kappa(x) = \kappa(y)$
	\end{theorem}
	\begin{proof}
		Misal $\kappa(x)\cap\kappa(y) \ne \emptyset$, akibatnya ada $a \in \kappa(x)\cap\kappa(y)$. Dari definisi kelas ekuivalen, haruslah $a \sim x$ dan $a \sim y$. Ambil $\xi \in \kappa(x)$, haruslah $\xi \sim x$. Karena $a \sim x$ dan $\xi \sim x$, dari sifat simetri dan transitif haruslah $a \sim \xi$. Karena $a \sim y$, akibatnya $\xi \sim y$ untuk setiap $\xi \in \kappa(x)$. Dengan demikian, $\kappa(x) \subseteq \kappa(y)$. Menggunakan argumen serupa, dapat ditunjukkan bahwa $\kappa(y) \subseteq \kappa(x)$ sehingga $\kappa(x) = \kappa(y)$.
	\end{proof}
	Teorema ini akan berguna ketika kita meninjau kelas-kelas ekuivalen dari suatu himpunan sebagai partisi atas himpunan tersebut. Kenapa? Karena dua kelas ekuivalen hanya akan beririsan jika dan hanya jika kedua kelas ekuivalen tersebut adalah kelas ekuivalen yang sama.
	\\
	
	\textbf{Definisi. }Misalkan $S$ adalah himpunan dan $P$ adalah himpunan yang berisi himpunan-himpunan yang merupakan subhimpunan tak hampa dari $S$. Kita katakan \textbf{$P$ partisi bagi $S$} (\textbf{$P$ mempartisi $S$}) jika: \begin{itemize}
		\item Gabungan dari setiap himpunan di $P$ adalah $S$
		\item Irisan dari setiap dua himpunan berbeda di $P$ adalah kosong, yakni untuk setiap himpunan $A,B \in P$, haruslah $(A \ne B) \implies A \cap B = \emptyset$
	\end{itemize}
	Dengan demikian, kita bisa dapatkan akibat dari teorema \ref{kardKlasEq}.
	\begin{corollary}
		Jika $\sim$ adalah relasi ekuivalen di $S$, kelas-kelas ekuivalen pada $S$ mempartisi $S$.
	\end{corollary}
	Dari teorema \ref{kardKlasEq}, kita sudah tahu bahwa poin kedua definisi partisi terpenuhi sehingga untuk membuktikan akibat tersebut, cukup ditunjukkan bahwa
	$$\bigcup_{s \in S} \kappa(s) = S$$
	Yakni gabungan dari setiap kelas ekuivalen di $S$ adalah $S$.
	\begin{proof}
		Jelas bahwa $\bigcup_{s \in S} \kappa(s) \subseteq S$. Ambil $x \in S$, tentu $x \sim x$. Jadi, $x \in \kappa(x)$ untuk setiap $x \in S$. Dengan demikian, $x \in \bigcup_{s \in S} \kappa(s)$ untuk setiap $x \in S$ sehingga $S \subseteq \bigcup_{s \in S} \kappa(s)$.
		\\
		
		Jadi, $\bigcup_{s \in S} \kappa(s) = S$. Dengan demikian, akibat tersebut terbukti.
	\end{proof}
	
	Misalkan himpunan $A$ terpartisi menjadi sebanyak \textit{countable} partisi (bisa saja tak berhingga, asalkan \textit{countable}). Definisikan $I$ sebagai himpunan indeks bagi partisi tersebut, yakni setiap partisi tersebut dapat dilabeli sebagai $A_i$ untuk suatu $i \in I$. Sebagai latihan, silahkan anda tunjukkan (atau cari penyangkal) bahwa untuk setiap $x,y \in A$, $x \sim y \iff x,y \in A_i$ untuk suatu $i \in I$ adalah suatu relasi ekuivalen. Yakni, tunjukkan bahwa untuk setiap $x,y \in A$, pernyataan "$x$ dan $y$ ekuivalen jika dan hanya jika ada pada himpunan partisi yang sama" mendefinisikan suatu relasi ekuivalen.
	\subsection{Modulo}
	Tinjau relasi di $\mathbb{Z}$ yang didefinisikan oleh $a \sim b \iff a - b \in n\mathbb{Z}$. Relasi inilah yang disebut sebagai relasi modulo $n$. Akan dibuktikan bahwa relasi modulo adalah relasi ekuivalen.
	\\
	
	Pertama, jelas bahwa untuk $a \in \mathbb{Z}$, $a - a = 0 \in n\mathbb{Z}$ sehingga $a \sim a$.
	\\
	
	Kemudian, untuk $a,b \in \mathbb{Z}$, jika $a \sim b$ berarti $a - b \in n\mathbb{Z}$. Karena $n\mathbb{Z}$ tertutup atas penjumlahan (dan pengurangan), $b - a \in \mathbb{Z}$ sehingga kita dapatkan $a \sim b \iff b \sim a$
	\\
	
	Terakhir, untuk $a,b,c \in \mathbb{Z}$, jika $a \sim b$ dan $b \sim c$, berarti $a - b $ dan $b - c$ di $n\mathbb{Z}$. Karena $n\mathbb{Z}$ tertutup atas penjumlahan, $(a - b) + (b - c)$ juga ada di $n\mathbb{Z}$ sehingga $a - c \in n\mathbb{Z}$. Jadi, $a \sim c$.
	\\
	
	Karena semua sifat relasi ekuivalen telah dibuktikan, kita simpulkan bahwa relasi modulo $n$ adalah sebuah relasi ekuivalen.
	\section{Kongruensi}
	Suatu relasi ekuivalen $\sim$ di himpunan $S$ yang dilengkapi operasi $\oplus$ dikatakan sebagai kongruensi jika dan hanya jika ia terdefinisi dengan baik. Yakni, untuk $a,a',b,b' \in S$, $a \sim a'$ dan $b \sim b'$ mengakibatkan $a \oplus b \sim a' \oplus b'$ .
	\\
	
	Tinjau kembali relasi modulo n yang telah didefinisikan pada subbab sebelumnya. Dapat ditunjukkan bahwa di sistem matematika $(\mathbb{Z},+)$, relasi tersebut merupakan sebuah kongruensi. Misalkan $\sim$ adalah relasi modulo n. Kita dapat notasikan $x \sim y$ sebagai $x \equiv y$ (mod $n$). Notasi ini dibaca sebagai \textbf{$x$ kongruen dengan $y$ modulo $n$.} Untuk setiap $a \in \mathbb{Z}$, definisikan pula
	$$\overline{a} = \{ x \in \mathbb{Z} \ | \ x \equiv a \ \textnormal{(mod }n) \}$$
	
	Jelas bahwa $\overline{a}$ adalah suatu kelas ekuivalen. Kita katakan bahwa $a$ adalah \textbf{wakil kelas} bagi kelas ekuivalen $\overline{a}$. Koleksi dari semua kelas ekuivalen dengan relasi ekuivalen modulo n adalah
	$$\mathbb{Z}_n = \{ \overline{a} \ \, | \, \ a \in \mathbb{Z}  \}$$
	Dapat diperiksa bahwa isi dari $\mathbb{Z}_n$ adalah $$\mathbb{Z}_n = \{\overline{0}, \overline{1}, ..., \overline{n-1} \}$$
	\\
	
	Tentunya untuk kasus ini berlaku $n > 1$.
	
	\section{Bilangan Bulat Modulo dan Propertinya}
	Definisikan suatu pengaitan
	\begin{equation*}
	\begin{split}
		& +: \mathbb{Z}_n \times \mathbb{Z}_n \rightarrow \mathbb{Z}_n\\
		& (\overline{a}, \overline{b}) \mapsto \overline{a+b}
	\end{split}
	\end{equation*}
	Akan dibuktikan bahwa + adalah suatu operasi di $\mathbb{Z}_n$. Untuk membuktikannya, cukup ditunjukkan bahwa operasi + terdefinisi dengan baik.
	Ambil $\overline{a_1},\overline{a_2},\overline{b_1},\overline{b_2} \in \mathbb{Z}_n$ dengan $\overline{a_1} = \overline{a_2}$ dan $\overline{b_1} = \overline{b_2}$. Akan ditunjukkan bahwa $\overline{a_1 + b_1} = \overline{a_2 + b_2}$.
	\\
	
	Dari definisi, kita dapatkan $a_1 - a_2 \in n\mathbb{Z}$ dan $b_1 - b_2 \in n\mathbb{Z}$. Karena $n\mathbb{Z}$ tertutup atas penjumlahan, kita dapatkan $(a_1 - a_2) + (b_1 - b_2) \in n\mathbb{Z}$. Berarti, $(a_1 + b_1) - (a_2 + b_2) \in n\mathbb{Z}$ sehingga $\overline{a_1 + b_1} = \overline{a_2 + b_2}$. Jadi, pengaitan + terdefinisi dengan baik. Kita simpulkan bahwa + adalah suatu operasi atas $\mathbb{Z}_n$. Karena + terdefinisi dengan baik, operasi + tidak bergantung pada wakil kelas.
	\\
	
	Dengan cara serupa, operasi $\times$ pada $\mathbb{Z}_n$ dapat didefinisikan, yaitu pemetaan yang memetakan $(\overline{a}, \overline{b})$ ke $\overline{ab}$. Pembaca diharap untuk memeriksa bahwa operasi tersebut terdefinisi dengan baik.
	\\
	
	Sekarang tinjau sistem matematika $(Z_n, +, \times)$ dengan operasi $+$ dan $\times$ yang telah didefinisikan di atas. Dapat diperiksa bahwa sistem matematika tersebut memenuhi sifat-sifat berikut:
	\begin{enumerate}
		\item $\overline{a} + \overline{b} = \overline{b} + \overline{a}$ untuk setiap $\overline{a}, \overline{b} \in \mathbb{Z}_n$
		\item $(\overline{a} + \overline{b}) + \overline{c} = \overline{a} + (\overline{b} + \overline{c})$ untuk setiap $\overline{a}, \overline{b}, \overline{c} \in \mathbb{Z}_n$
		\item Ada unsur $\overline{0} \in \mathbb{Z}_n$ sehingga untuk setiap $\overline{a} \in \mathbb{Z}_n$ berlaku $\overline{a} + \overline{0} = \overline{0} + \overline{a} = \overline{a}$
		\item Untuk setiap $\overline{a} \in \mathbb{Z}_n$ ada unsur $-\overline{a} \in \mathbb{Z}_n$ sehingga $\overline{a} + (-\overline{a}) = (-\overline{a}) + \overline{a} = \overline{0}$
		\item $\overline{a} \overline{b} = \overline{b} \overline{a}$ untuk setiap $\overline{a}, \overline{b} \in \mathbb{Z}_n$
		\item $\overline{a}(\overline{b}\overline{c}) = (\overline{a} \overline{b})\overline{c}$ untuk setiap $\overline{a},\overline{b},\overline{c} \in \mathbb{Z}_n$
		\item Ada $\overline{1} \in \mathbb{Z}_n$ sehingga untuk setiap $\overline{a} \in \mathbb{Z}_n$ berlaku $\overline{1}\overline{a} = \overline{a}\overline{1} = \overline{a}$
		\item $(\overline{a} + \overline{b})\overline{c} = \overline{a}\overline{c} + \overline{b}\overline{c}$ untuk setiap $\overline{a},\overline{b},\overline{c} \in \mathbb{Z}_n$
	\end{enumerate}
	Bukti untuk pernyataan-pernyataan tersebut dapat dilakukan dengan menggunakan sifat-sifat modulo. Sebagai contoh, akan dibuktikan sifat (4).
	\\
	
	\begin{proof}[Eksistensi Invers Penjumlahan]
		Ambil $\overline{a} \in \mathbb{Z}_n$. Karena $\overline{a}$ diambil bebas, dapat dipilih nilai $a$ yang memenuhi $0 \le a \le n-1$.
		
		Jika $a = 0$, dapat dipilih $-\overline{a} = 0$. Jika $a \ne 0$, pilih $b = n-a$. Jelas bahwa $1 \le b \le n-1$ sehingga $\overline{b} \in \mathbb{Z}_n$ dan $\overline{b} \ne 0$. Perhatikan bahwa
		\begin{equation*}
		\begin{split}
			\overline{a} + \overline{b} & = \overline{a} + \overline{n - a}\\
			& = \overline{a + n - a}\\
			& = \overline{n}\\
			& = \overline{0}
		\end{split}
		\end{equation*}
		Jadi kita dapatkan $\overline{b} = -\overline{a}$ untuk setiap $\overline{a} \ne 0 \in \mathbb{Z}_n$. Dengan demikian, untuk setiap $\overline{a} \in \mathbb{Z}_n$ ada $-\overline{a}$ sehingga $\overline{a} + (-\overline{a}) = \overline{0}$
	\end{proof}
	Properti-properti ini serupa dengan properti $(\mathbb{Z},+,\times)$ yang telah kita bahas sebelumnya. Kita dapatkan $\mathbb{Z}_n$ adalah gelanggang dengan operasi-operasi yang didefinisikan di atas.
	\\
	
	\section{Fungsi Phi Euler}
	Definisikan himpunan $U_n$ = $\{ \overline{k} \in \mathbb{Z}_n \ | \ (k,n) = 1 \}$. Tentunya $k$ dapat dipilih agar memenuhi $0 \le k \le n-1$ sehingga himpunan ini berisi semua bilangan bulat positif lebih kecil dari $n$ yang relatif prima dengan $n$. Kardinalitas dari $U_n$, yaitu $|U_n|$ cukup sering muncul dalam bahasan terkait teori bilangan dan kita beri nama.
	\\
	
	\textbf{Definisi. } Fungsi Phi Euler, $\phi(n)$, didefinisikan dengan $\phi(1) = 1$ dan $\phi(n) = $ banyaknya bilangan antara 1 dan $n-1$ (inklusif) yang relatif prima dengan $n$.
	\\
	
	Sebagai contoh, $\phi(2) = 1$, $\phi(8) = 4$, dan $\phi(19) = 18$. Dapat dengan mudah dilihat bahwa untuk $p$ prima, $\phi(p) = p-1$ (silahkan periksa).
	\\
	
	Sebagai contoh penggunaannya pada teori bilangan, terdapat teorema dari Euler.
	
	\begin{theorem}[Teorema Euler]
		Misalkan $a,n \in \mathbb{N}$ dengan $(a,n) = 1$. Kita dapatkan $a^{\phi(n)} \equiv 1 \ (\textnormal{mod } n)$
	\end{theorem}
	Bukti dari teorema ini di luar cakupan bahasan kuliah struktur bilangan bulat. Pembaca yang ingin tahu lebih lanjut dapat membaca buku \textit{Abstract Algebra} karya I.N. Herstein bab 2.4 (teorema Lagrange). Konstruksi bukti akan melibatkan orde unsur pada grup yang dibentuk himpunan $U_n$.
	\\
	
	Kasus khusus untuk teorema ini adalah ketika $n = p$ prima, yang dikenal sebagai teorema kecil Fermat (\textit{Fermat Little Theorem})
	
	\begin{corollary}[Teorema Kecil Fermat]
		Jika $p$ prima dan $p$ tidak membagi $a$, $a^{p-1} \equiv 1 \ (\textnormal{mod } p)$ 
	\end{corollary}
	\begin{proof}
		Karena $\phi(p) = p-1$ untuk $p$ prima, kita gunakan teorema Euler untuk mendapatkan akibat tersebut.
	\end{proof}
	Kita juga dapatkan akibat berikutnya
	
	\begin{corollary}
		Misalkan $p$ prima. Untuk sembarang bilangan bulat $b$, $b^p \equiv b \ (\textnormal{mod } p)$
	\end{corollary}
	\begin{proof}
		Andaikan $p$ tidak membagi $b$, kita gunakan teorema kecil Fermat dan kalikan dengan $b$ untuk mendapat hasil yang diinginkan. Untuk $p$ habis membagi $b$, $b \equiv 0 \ (\textnormal{mod }p)$ sehingga $b^p \equiv 0 \ (\textnormal{mod }p)$. Akibatnya, $b^p \equiv b \ (\textnormal{mod }p)$
	\end{proof}

	Tinjau kembali sistem matematika $(\mathbb{Z}_n, +, \times)$. Kita akan punya hasil berikut.
	
	\begin{theorem}
		Setiap $\overline{a} \in \mathbb{Z}_n$ dengan $a$ relatif prima dengan $n$ mempunyai invers perkalian
	\end{theorem}
	\begin{proof}
	Jalannya bukti akan memanfaatkan teorema Euler. Jika $(a,n) = 1$, dari teorema Euler kita dapatkan $\overline{a^{\phi{(n)}}} = \overline{1}$. Dengan kata lain, 
	\begin{equation*}
	\begin{split}
		\overline{a}\overline{a^{\phi(n)-1}} & = \overline{a a^{\phi(n)-1}}\\
		& = \overline{a^{\phi(n)}} \\
		& = \overline{1}	 
	\end{split}
	\end{equation*}
	Sehingga $\overline{a^{-1}} = \overline{a^{\phi(n-1)}}$ invers bagi $\overline{a}$
	\end{proof}
	
	Akibat dari hasil ini adalah untuk $n = p$ prima, setiap unsur di $\mathbb{Z}_p$ yang tak nol mempunyai invers perkalian. Struktur seperti ini akan dikenal sebagai \textbf{lapangan} dalam studi aljabar abstrak atau aljabar linear (sebagai skalar dari suatu ruang vektor).
	
\chapter{Gelanggang}
	\section{Gelanggang dan Propertinya}
	Kali ini kita akan meninjau secara umum struktur dengan properti yang telah kita kenali sebelumnya. Tinjau kembali sistem matematika ($\mathbb{Z}_n,+,\times$). Properti-properti yang dimiliki (secara singkat) adalah komutatif pada operasi penjumlahan, asosiatif untuk operasi penjumlahan dan perkalian, eksistensi unsur identitas untuk operasi penjumlahan dan perkalian (unsur $0$ dan $1$), eksistensi invers penjumlahan, serta sifat distributif. Kali ini, kita akan meninjau sistem matematika lain dengan properti serupa. Setiap sistem matematika dengan properti tersebut akan kita sebut sebagai \textbf{gelanggang} atau \textit{ring}.
	\\
	
	Kita mulai dengan mendefinisikan gelanggang.
	\\
	
	\textbf{Definisi. }Suatu sistem matematika dengan dua operasi $(R,+,\cdot)$ (operasi pertama disebut sebagai penjumlahan dan kedua sebagai perkalian) merupakan \textbf{gelanggang} (dengan unsur kesatuan) jika untuk $a,b,c \in R$ berlaku
	\begin{enumerate}
		\item $a + b = b + a \in R$ (komutatif dan tertutup atas penjumlahan)
		\item $a + (b + c) = (a + b) + c$ (asosiatif penjumlahan)
		\item Ada $0 \in R$ sehingga $0 + a = a + 0 = a$
		\item Ada $-a \in R$ sehingga $a + (-a) = 0 = (-a) + a$
		\item Ada $1 \ne 0 \in R$ sehingga $a \cdot 1 = 1 \cdot a = a$
		\item $a \cdot (b \cdot c) = (a \cdot b) \cdot c$ (asosiatif perkalian)
		\item $a \cdot (b + c) = a \cdot b + a \cdot c$ (distributif)
	\end{enumerate}
	(coba anda periksa juga apakah $n\mathbb{Z}$ merupakan gelanggang).\\
	
	Tentunya $R$ harus tertutup atas $+$ dan $\cdot$ karena $+$ dan $\cdot$ adalah operasi di $R$. Beberapa literatur ada yang tidak mensyaratkan $1 \ne 0$ (sehingga identitas perkalian dan penjumlahan sama) untuk gelanggang, oleh sebab itu pada pendefinisian di atas penulis menyebut definisi tersebut sebagai gelanggang dengan unsur kesatuan (yaitu 1). Berikutnya struktur seperti ini hanya akan kita sebut sebagai gelanggang saja. Sebagai catatan, umumnya unsur 0 (identitas penjumlahan) disebut sebagai unsur identitas (saja) dan unsur 1 (identitas perkalian) disebut sebagai unsur kesatuan.
	\\
	
	Hati-hati dengan operasi perkalian karena definisi gelanggang tidak mensyaratkan kekomutativannya dan eksistensi inversnya. Jika suatu gelanggang operasi perkaliannya juga komutatif, kita sebut sebagai \textbf{gelanggang komutatif}. Beberapa contoh gelanggang adalah (silahkan anda periksa yang mana saja yang komutatif dan jika perlu, buktikan) \begin{enumerate}
		\item $\mathbb{Z}$ atas operasi biasa
		\item $\mathbb{Q}$ atas operasi biasa
		\item $\mathbb{C}$ atas operasi biasa
		\item $M_{nn}(\mathbb{R})$, himpunan matriks $n \times n$ dengan koefisien real, atas operasi (matriks) biasa
		\item $\mathbb{Z}_n$ atas operasi modulo
	\end{enumerate}
	
	Berikutnya, kita akan definisikan sesuatu yang disebut sebagai pembagi nol.\\
	
	\textbf{Definisi. }Untuk $R$ suatu gelanggang dan $a \in R$ dengan $a \ne 0$, $a$ dikatakan sebagai \textbf{pembagi nol} jika ada $b \in R$ tak nol sehingga $ab = 0$.
	\\
	
	Tidak semua gelanggang punya pembagi nol. Sebagai contoh, gelanggang $\mathbb{Z}$ atas operasi biasa tidak memiliki pembagi nol. Untuk contoh pembagi nol, di $\mathbb{Z}_6$ berlaku $\overline{2}\times \overline{3} = \overline{0}$. Contoh lainnya, silahkan anda coba cari semua pembagi nol di himpunan matriks $n\times n$ dengan koefisien real. (Hint: ingat kembali kuliah aljabar linier anda.)
	\\
	
	Sekarang, gelanggang-gelanggang komutatif yang tidak memiliki pembagi nol akan kita beri nama khusus, yaitu \textbf{daerah integral} (\textit{integral domain}). Bedakan istilah "daerah integral" di sini dengan daerah pengintegralan pada kalkulus. Kata "integral" di sini merujuk ke "\textit{integer}" (dapat anda periksa bahwa gelanggang bilangan bulat memang merupakan daerah integral).
	\\
	
	Pada daerah integral, kita punya teorema yang sudah kita kenal berikut:
	
	\begin{theorem}[Pencoretan]
	Misal $D$ suatu daerah integral dan $x,y,z \in D$. Untuk $x \ne 0$ kita punya $xy = xz$ mengakibatkan $y = z$.
	\end{theorem}
	\begin{proof}
		Perhatikan bahwa 
		\begin{equation*}
		\begin{split}
		xy &= xz \\
		\iff xy - xz &= 0\\
		\iff x(y + (-z)) &= 0\\
		\end{split}
		\end{equation*}
		hanya dipenuhi ketika $x = 0$ atau $y + (-z) = 0$. Karena $x \ne 0$, haruslah $y + (-z) = 0$. Menjumlahkan kedua ruas dengan $z$, kita dapat $y = z$.
	\end{proof}
	Kemudian, kita bahas satu struktur lagi yang banyak muncul pada bahasan aljabar linier (terutama terkait ruang vektor), yaitu struktur lapangan.
	\\
	
	\textbf{Definisi.} Suatu sistem matematika $(F, +, \times)$ adalah suatu \textbf{lapangan} (atau dalam bahasa Inggris disebut \textit{field}) jika: \begin{enumerate}
		\item ($F,+,\times$) adalah sebuah gelanggang komutatif
		\item ($F$\textbackslash $\{0\},\times)$ adalah sebuah grup komutatif (biasanya disebut sebagai grup perkalian dari lapangan $F$)
	\end{enumerate}
	Dalam bahasa yang lebih manusiawi, suatu lapangan adalah gelanggang komutatif yang setiap unsur tak nolnya punya invers perkalian.
	\\
	
	Hubungan antara daerah integral dengan lapangan dibahas dalam dua teorema berikut:
	
	\begin{theorem}
	Setiap lapangan adalah daerah integral
	\end{theorem}
	\begin{proof}
		Misal $F$ suatu lapangan. Ambil unsur tak nol di $F$, katakan $x \in F$. Andaikan $F$ bukan suatu daerah integral, ada $y \ne 0$ sehingga $xy = 0$. Akibatnya, \begin{equation*}
		\begin{split}
			x^{-1}(xy) &= (x^{-1})0 \\
			(x^{-1}x)y &= (x^{-1})0 \\
			y &= 0
		\end{split}
		\end{equation*}
		Kontradiksi dengan $y \ne 0$. Jadi, haruslah $F$ suatu daerah integral.
	\end{proof}
	Apakah sebaliknya berlaku? Yakni, apakah setiap daerah integral adalah lapangan? Jika tidak, cobalah anda cari contoh daerah integral yang bukan lapangan.\\
	
	Teorema berikut ini membahas kasus khusus ketika suatu daerah integral juga merupakan suatu lapangan.
	\begin{theorem}
		Setiap daerah integral dengan banyaknya unsur berhingga adalah suatu lapangan
	\end{theorem}
	\begin{proof}
		Misal $D$ suatu daerah integral dengan banyak unsur berhingga dan $x \ne 0 \in D$. Kita perlu tunjukkan sebarang $x$ tak nol tersebut punya invers. Tinjau himpunan $S = \{x, x^2, x^3, ... \}$. Jelas $S \subseteq D$. Karena $D$ punya berhingga unsur, haruslah $S$ punya berhingga unsur. Akibatnya, ada $m,n$ dengan $n < m$ sehingga $x^{m} = x^{n}$ sehingga $x^{m-n}x^{n} = x^{n}$.\\
		
		Sekarang, andaikan $m - n = 1$, kita dapatkan $x = 1$, jelas $x$ punya invers. Andaikan tidak, kita punya $x^{m-n} = x(x^{m-n-1})$. Karena $x^m = x^n$, jelas $x^{m-n} = 1$ sehingga $x(x^{m-n-1}) = 1$.\\
		
		Perhatikan bahwa karena $D$ daerah integral, operasi perkalian bersifat komutatif. Jadi, $x(x^{m-n-1}) = 1 = (x^{m-n-1})x$. Berdasarkan definisi invers, kita dapatkan $x^{m-n-1}$ adalah invers dari $x$. Karena $x$ dipilih sebarang asal tak nol, setiap unsur $x \in D$ tak nol punya invers. Jadi, $D$ adalah suatu lapangan.
	\end{proof}
	Lapangan dapat juga didefinisikan dalam daerah integral. Salah satu definisi lain lapangan adalah daerah integral yang setiap unsur tak nolnya punya invers. Silahkan anda periksa bahwa definisi ini dan definisi lapangan sebelumnya ekuivalen.
	\\
	
	Karena ini adalah catatan terkait struktur bilangan bulat, marilah kembali ke gelanggang bilangan bulat modulo. Pada gelanggang $\mathbb{Z}_n$, kita punya teorema berikut.
	
	\begin{theorem}
		Misal $\overline{a} \in \mathbb{Z}_n$ tak nol. Unsur $\overline{a}$ punya invers jika dan hanya jika $GCD(a,n)=1$.
	\end{theorem}
	\begin{proof}
		Misalkan GCD($a,n$) $ = 1$. Dari teorema Euler, kita punya $\overline{a}^{\phi(n)} = 1$. Berarti, $\overline{a}\overline{a}^{\phi(n)-1} = \overline{1}$. Akibatnya, $\overline{a}$ punya invers.
		\\
		
		Sebaliknya, misal $\overline{a}$ punya invers. Tulis $\overline{a}\overline{b} = 1$ untuk suatu $\overline{b} \in \mathbb{Z}_n$. Dari pendefinisian unsur di $\mathbb{Z}_n$, ada $k,l,j \in \mathbb{Z}$ sehingga
		\begin{equation*}
		\begin{split}
		(kn+a)(ln+b) &= jn+1\\
		\iff n(kln + bk - j) + a(ln + b) &= 1
		\end{split}
		\end{equation*}
		Persamaan tersebut dapat kita tulis secara sederhana sebagai $$np + aq = 1$$
		untuk $p,q \in \mathbb{Z}$.
		Akibatnya, GCD($a,n$) = $1$.
	\end{proof}
	\section{Ideal}
	\subsection{Pengenalan}
	Salah satu bahasan penting ketika mengkaji teori gelanggang adalah ideal. Misal $(R,+,\times)$ gelanggang. Suatu subhimpunan tak hampa $I \subseteq R$ dikatakan \textbf{ideal kiri} (atau \textbf{ideal kanan}) dari gelanggang $R$ jika:
	\begin{enumerate}
		\item $(I,+)$ adalah subgrup dari $(R,+)$
		\item Untuk setiap $x \in I$ dan $r \in R$ berlaku $rx \in I$ (atau $xr \in I$ jika ideal kanan)
	\end{enumerate}
	Jika cukup jelas ideal kiri atau ideal kanan (atau jika suatu ideal adalah ideal kiri dan ideal kanan), umumnya hanya disebut sebagai ideal saja (atau ideal 2 sisi jika ideal kanan-kiri). 
	
	Sebagai contoh, pada gelanggang $M_{2\times 2}(\mathbb{Z})$, himpunan $$I = \left\{ \left( \begin{matrix}
	a & 0\\
	b & 0\\
	\end{matrix}\right) \, | \, a,b \in \mathbb{Z} \right\}$$ adalah suatu ideal kiri, tetapi bukan ideal kanan (silahkan periksa). Untuk alasan praktis, berikutnya ideal di sini akan disebut sebagai ideal saja, tanpa menyebutkan ideal kiri atau ideal kanan. Untuk ideal yang bukan dua sisi, asumsikan sebagai ideal kiri.
	
	Untuk memeriksa apakah suatu subhimpunan merupakan ideal atau bukan, dapat digunakan teorema berikut untuk mempermudah
	
	\begin{theorem}
		Misal $I \subseteq R$ tak hampa. Kita katakan $I$ adalah ideal jika dan hanya jika
		\begin{enumerate}
			\item Untuk setiap $a,b \in I$ berlaku $a+b \in I$
			\item Untuk setiap $r \in R$ dan $x \in I$ berlaku $rx \in I$
		\end{enumerate}
	\end{theorem}
	\begin{proof}
		Arah kanan pernyataan tersebut jelas. Sekarang, untuk arah sebaliknya, perhatikan bahwa karena $R$ suatu gelanggang, jelas $0 \in R$. Jadi, kita dapatkan $0 \times x = 0 \in I$. Karena $R$ gelanggang, jelas $1$ dan $-1$ di $R$. Ambil sebarang $x \in I$. Kita punya $-x \in I$ sehingga $(I,+)$ adalah subgrup $(R,+)$. Kita simpulkan $I$ adalah ideal dari $R$.
	\end{proof}

	Berikutnya kita akan tinjau gelanggang $\mathbb{Z}$. Salah satu ideal dari gelanggang ini adalah ideal nol, ideal yang isinya seluruh $\mathbb{Z}$. Adakah ideal lain dari gelanggang ini?
	Dugaan yang cukup wajar adalah ideal dari $\mathbb{Z}$ berbentuk $n\mathbb{Z}$. Apakah benar ini ideal? Apakah semua idealnya berbentuk seperti itu? Kita punya teorema berikut
	
	\begin{theorem}
		$I$ adalah ideal dari $\mathbb{Z}$ jika dan hanya jika $I = n\mathbb{Z}$ untuk suatu $n \in \mathbb{Z}$
	\end{theorem}
	\begin{proof}
		Kita tinjau arah kanan teorema tersebut lebih dahulu. Misal $I = n\mathbb{Z}$ untuk suatu $n \in \mathbb{Z}$. Ambil $x,y \in I, r \in \mathbb{Z}$, akan ada $k,l \in \mathbb{Z}$ sehingga $x = kn$ dan $y = ln$. Akibatnya,
		$$x+y = n(k+l) \in n\mathbb{Z}$$
		$$rx = r(kn) = n(rk) \in n\mathbb{Z}$$
		Jadi $I = n\mathbb{Z}$ adalah suatu ideal.
		\\
		
		Sebaliknya, misal $I$ suatu ideal dari $\mathbb{Z}$. Untuk $I = \{0\}$ atau $I = \mathbb{Z}$, jelas $I = 0\mathbb{Z}$ atau $I = 1\mathbb{Z}$. Untuk ideal lainnya, pilih $n$ sebagai suatu unsur positif terkecil di $I$. Akibatnya, $n\mathbb{Z} \subseteq I$. Berikutnya, ambil sebarang $x \in I$. Dengan algoritma pembagian, tulis
		$$x = pn + q$$
		untuk suatu $p,q \in \mathbb{Z}$ dengan $0 \le q \le n-1$. Karena $x,n \in I$ haruslah $x - pn \in I$ sehingga haruslah juga $q \in I$. Perhatikan bahwa karena $n$ unsur positif terkecil di $I$ dan $0 \le q \le n-1$, haruslah $q = 0$. Jadi, $x = pn$ untuk suatu $p \in I$. Akibatnya, $I \subseteq n\mathbb{Z}$. Karena kedua himpunan saling subset, kita dapatkan $I = n\mathbb{Z}$.
	\end{proof}
	Satu hal penting yang perlu diperhatikan dari ideal adalah teorema berikut
	\begin{theorem}
		Misal $I \subseteq R$ suatu ideal. Kesamaan $I = R$ terjadi jika dan hanya jika $1 \in I$
	\end{theorem}
	\begin{proof}
		Andaikan $I = R$, jelas $1 \in I$. Sebaliknya, misal $1 \in I$. Untuk sebarang $r \in R$, $r \times 1  = r \in I$ sehingga $R \subseteq I$. Jelas $I \subseteq R$ sehingga $I = R$.
	\end{proof}
	Akibat dari teorema di atas adalah
	\begin{corollary}
		Misal $F$ suatu lapangan. $I$ ideal dari $F$ jika dan hanya jika $I = \{0 \}$ atau $I = F$.
	\end{corollary}
	\hfill
	
	Pendefinisian berikutnya akan memanfaatkan pengertian terkait ideal sejati (\textit{proper ideal}). Suatu ideal $I$ dari gelanggang $R$ dikatakan \textbf{ideal sejati} jika $I \ne R$.\\
	
	\textbf{Definisi. }Suatu ideal sejati $I$ dari $R$ yang maksimal dikatakan sebagai \textbf{ideal maksimal} dari $R$. Dengan kata lain, $I$ adalah ideal maksimal dari gelanggang $R$ jika untuk setiap ideal $J$ dengan $I \subset J$ berlaku $J = R$.\\
	
	Perhatikan bahwa pendefinisian tersebut tidak menjamin ketunggalan dari ideal maksimal. Sebagai contoh, ideal $2\mathbb{Z}$ dan $3\mathbb{Z}$ sama-sama merupakan ideal maksimal dari gelanggang $\mathbb{Z}$ (silahkan anda periksa).
	\\
	
	\textbf{Diskusi.} Apakah $I = \{ x(2+i) \, \, | \, \, x \in \mathbb{Z}[i] \}$ adalah ideal maksimal di $\mathbb{Z}[i]$?
	\subsection{Daerah Ideal Utama}
	Pada bagian ini, kita akan mengkaji gelanggang dengan properti khusus. Sebelumnya, kita definisikan dulu apa itu ideal utama dan unsur pembangun. Suatu ideal (kiri) $I$ dikatakan \textbf{dibangun} oleh unsur $n \in I$ jika $I = \{rn \, | \, r \in R\}$. Notasikan dengan $I = \langle n \rangle$. Ideal yang dibangun oleh tepat satu unsur kemudian akan kita sebut sebagai suatu \textbf{ideal utama}.
	\\
	
	Untuk daerah integral yang setiap idealnya adalah ideal utama, kita katakan sebagai \textbf{daerah ideal utama}. Perhatikan bahwa untuk $D$ suatu daerah integral, $D = \langle 1 \rangle$.\\
	
	Contoh gelanggang yang bukan daerah ideal utama adalah gelanggang quarternion $(\mathbb{H},+,\times)$. Gelanggang quarternion adalah perluasan dari bilangan kompleks dengan 4 dimensi. Untuk $x \in \mathbb{H}$, dapat ditulis $x = a + ib + cj + dk$ (serupa dengan pada bilangan kompleks $z = x + iy$) dengan $i^2 = j^2 = k^2 = -1$, $ij = k, jk = i,$ dan $ki = j$. Perhatikan bahwa $I = \langle i,j \rangle = \{ri + kj \, | \, r,k \in \mathbb{H}\}$ adalah ideal (dan bukan ideal utama). Namun, selain melanggar syarat ideal utama, quarternion bahkan bukan gelanggang komutatif, sehingga bukan daerah integral. Akibatnya, jelas $\mathbb{H}$ bukan daerah ideal utama.
	\\
	
	Pada daerah ideal utama, kita punya teorema berikut
	\begin{theorem}[\textit{Ascending Chain Condition}]
	Misal $D$ suatu daerah ideal utama dan $$\mathcal{C} := I_1 \subseteq I_2 \subseteq I_3 \subseteq ...$$ suatu rantai naik ideal dari $D$. Ada $k \ge 1$ sehingga $I_k = I_{k+j}$ untuk setiap $j \ge 0$
	\end{theorem}
	\begin{proof}
		Tulis $$I = \bigcup_{I_l \in \mathcal{C}} I_l$$
		Ambil $a,b \in I$, jelas ada $r_1, r_2$ sehingga $a \in I_{r_1}$ dan $b \in r_2$. Tanpa mengurangi keumuman, pilih $r_1 \le r_2$ (perhatikan bahwa bisa saja $r_1 = r_2$, tidak masalah). Dari rantai naik ideal, kita simpulkan $a,b \in I_{r_2}$. Karena $I_{r_2}$ ideal, jelas untuk sebarang $x_1, x_2 \in D$ berlaku $ax_1 + bx_2 \in I_{r_2}$. Dengan demikian, dapat kita simpulkan bahwa $I$ adalah suatu ideal. Karena $D$ suatu daerah ideal utama, ada $y \in D$ sehingga $I = \langle y \rangle$. Dari pendefinisian $I$, ada suatu $k$ sehingga $y \in I_k$.
		\\
		
		Berikutnya, akan ditunjukkan bahwa $I = I_k$. Jelas bahwa $I_k \subseteq I$. Ambil sebarang $z \in I$. Karena $I = \langle y \rangle$, ada $m \in D$ sehingga $z = my$. Jadi, $I \subseteq I_k$. Dengan demikian, kita simpulkan $I = I_k$
		\\
		
		Langkah terakhir adalah menunjukkan bahwa $I_k = I_{k+j}$ untuk sebarang $j \ge 0$. Dari rantai naik ideal, jelas $I_k \subseteq I_{k+j}$. Dari pendefinisian $I$, kita dapatkan pula $I_{k+j} \subseteq I$. Karena $I_k = I$, kita simpulkan bahwa $I_k = I_{k+j}$ untuk sebarang $j \ge 0$. Dengan demikian kita selesai.
	\end{proof}
	Sebelum lanjut ke teorema berikutnya, akan dikenalkan dahulu tentang lema Zorn. Materi ini sebenarnya di luar cakupan kuliah struktur bilangan bulat. Namun, teorema berikutnya adalah topik yang cukup penting.\\
	
	\textbf{Lema Zorn.} Misal $P$ adalah suatu himpunan yang dilengkapi urutan parsial (poset) dengan setiap rantai pada $P$ memiliki batas atas di $P$. Haruslah $P$ punya setidaknya satu batas atas.\\
	
	Detail terkait lema Zorn tidak akan dibahas di sini. Lema Zorn tersebut kemudian akan kita buktikan untuk membuktikan teorema berikut:
	
	\begin{theorem}
		Daerah ideal utama $D$ punya setidaknya satu ideal maksimal
	\end{theorem}

	\begin{proof}
		Definisikan $P = \{ I \subseteq D \ | \ I \ \textnormal{ideal}, I \ne D  \}$, yaitu himpunan dari ideal-ideal sejati $D$. Definisikan urutan parsial pada $P$ sebagai inklusi himpunan $(\subseteq)$. Definisikan rantai naik ideal $\mathcal{C}$ sebagai sebarang rantai naik ideal di $P$. Tulis $$I = \bigcup_{I_l \in \mathcal{C}} I_l$$
		Pada teorema \textit{ascending chain condition}, telah ditunjukkan bahwa $I$ adalah batas atas pada rantai $\mathcal{C}$ dan $I \in \mathcal{C}$. Agar memenuhi kondisi lema Zorn, perlu ditunjukkan bahwa $I \in P$, yakni $I \ne D$. Telah ditunjukkan sebelumnya bahwa $I$ adalah suatu ideal. Karena $1 \not\in P$ (kenapa?), akibatnya untuk sebarang $I_j \in P$, $1 \not\in I_j$. Jadi, $1 \not\in I$. Kita simpulkan $I \ne D$. Jadi, batas atas untuk sebarang rantai naik $\mathcal{C}$ di $P$ ada di $P$. Akibatnya, haruslah $P$ mempunyai batas atas. Batas atas ini adalah ideal maksimal dari $D$, sehingga $D$ punya ideal maksimal.
	\end{proof}
	Kita tahu bahwa ideal-ideal dari suatu daerah ideal utama dapat dibangun oleh tepat satu unsur. Lantas, unsur apakah yang membangun suatu ideal tersebut? Misal $A = \{ a_1, a_2, ..., a_n \}$ sehingga $I = \langle A \rangle$, dapatkah kita mencari satu unsur lain, katakan saja $d$, sehingga $I = \langle d \rangle$? Ternyata bisa. Kita punyai teorema berikut untuk menjawabnya.
	\begin{theorem}
		\label{generatorIU}
		Misal $D$ suatu daerah ideal utama dan $I$ ideal dari $D$ dengan $I = \langle A \rangle$. Ideal $I$ adalah ideal utama yang dibangun oleh $d$ jika dan hanya jika $d = $ GCD($A$)
	\end{theorem}
	\begin{proof}
		($\implies$) Misal $I = \langle d \rangle$. Karena $I = \langle A \rangle$, haruslah $a \in \langle d \rangle$ untuk setiap $a \in A$. Berarti, kita bisa tulis $a = pd$ untuk suatu $p \in I$. Akibatnya, $d | a$. Misalkan ada $c \in D$ sehingga $c|a$ untuk setiap $a \in A$. Karena $I$ dibangun $A$, kita bisa tulis $$d = \sum_{a_i \in A,\, k_i \in D} k_i a_i$$Akibatnya, $c | d$. Jadi, $d = $ GCD($A$).
		\\
		
		($\impliedby$) Misal $d = $ GCD($A$), haruslah $d | a$ untuk setiap $a \in A$. Akibatnya, haruslah $d \in I$ sehingga $\langle d \rangle \subseteq I$. Ambil sebarang $x \in I$. Karena $I = \langle A \rangle$, kita bisa tulis $$x = \sum_{a_i \in A, \, k_i \in D} k_i a_i$$Karena $d | a$, haruslah $d | x$ sehingga ada $p$ yang membuat $x = pd$. Akibatnya, $x \in \langle d \rangle$ sehingga $I \subseteq \langle d \rangle$. Dengan demikian, $I = \langle d \rangle$.
	\end{proof}
	\subsection{Ideal Prima}
	Motivasi dari pendefinisian ideal prima serupa dengan pendefinisian unsur prima pada suatu gelanggang. Sebagai pengingat, unsur $p$ dikatakan sebagai unsur prima jika $p$ bukan nol, $p$ bukan unit, dan $p | ab$ mengakibatkan $p | a$ atau $p | b$.\\
	
	\textbf{Definisi.} Suatu ideal sejati $I$ dari gelanggang $R$ dikatakan \textbf{ideal prima} jika  untuk $a,b \in R$, $ab \in I$ mengakibatkan $a \in I$ atau $b \in I$.
	\\
	
	Hubungan antara unsur prima di gelanggang bilangan bulat dengan ideal prima disajikan dalam teorema berikut:
	
	\begin{theorem}
		\label{id-prime}
		Unsur $p \in \mathbb{Z}$ adalah unsur prima jika dan hanya jika $p\mathbb{Z}$ adalah suatu ideal prima yang bukan ideal nol
	\end{theorem}
	\begin{proof}
		($\implies$) Misal $p$ prima. Ambil $x \in p\mathbb{Z}$. Andaikan $x$ prima, kita dapat tulis $x = ac$ untuk suatu $a$ unit dan $c$ prima. Karena $x \in p\mathbb{Z}$, haruslah $c = p$ (mengapa?). Jadi, $x = ap \in p\mathbb{Z}$. Jelas $p \in p\mathbb{Z}$. Kemudian, untuk $x$ tak prima, karena $x \in p\mathbb{Z}$ kita dapat tulis $x = dp$ untuk suatu $d \in \mathbb{Z}$. Jelas $p \in \mathbb{Z}$. Kita simpulkan bahwa $p\mathbb{Z}$ adalah ideal prima.\\
		
		($\impliedby$) Misal $p\mathbb{Z}$ suatu ideal prima. Untuk sebarang $a,b \in \mathbb{Z}$ dengan $ab \in p\mathbb{Z}$ berlaku $a \in p\mathbb{Z}$ atau $b \in p\mathbb{Z}$. Tanpa mengurangi keumuman, $a \in p\mathbb{Z}$. Berarti, untuk $ab \in p\mathbb{Z}$ berlaku $a = pk$ untuk suatu $k \in \mathbb{Z}$. Karena $ab \in p\mathbb{Z}$, kita bisa tulis $ab = pc$ untuk suatu $c \in \mathbb{Z}$. Dari definisi keterbagian, kita katakan $p | ab$ dan $p|a$. Apabila $b \in p\mathbb{Z}$, kita akan dapatkan $p | b$. Jadi, $p$ adalah unsur prima.
	\end{proof}
	Berikutnya, suatu observasi penting adalah ideal nol belum tentu merupakan ideal prima dari suatu gelanggang. Kapan ideal nol adalah ideal prima? Jawabannya ada dalam pendefinisian pembagi nol dan daerah integral. Kita dapatkan teorema berikut:
	
	\begin{theorem}
	Ideal nol adalah ideal prima dari gelanggang komutatif $D$ jika dan hanya jika $D$ adalah suatu daerah integral
	\end{theorem} 
	\begin{proof}
		($\implies$) Misal $I$ ideal nol adalah ideal prima, untuk sebarang $ab \in I$ berlaku $a \in I$ atau $b \in I$. Berarti $ab = 0$ jika dan hanya jika $a = 0$ atau $b = 0$. Akibatnya, haruslah $D$ suatu daerah integral.
		\\
		
		($\impliedby$) Misal $D$ adalah suatu daerah integral dan $I$ ideal nol. Untuk sebarang $x,y \in D$ dengan $xy = 0$, haruslah $x = 0$ atau $y = 0$. Akibatnya, untuk $xy \in I$ haruslah $x = 0 \in I$ atau $y = 0 \in I$. Jadi, $I$ adalah ideal prima.
	\end{proof}
	Berikutnya, untuk ideal $I$ tak nol dari gelanggang bilangan bulat $\mathbb{Z}$ pernyataan berikut ekuivalen (yakni semuanya benar atau semuanya salah):
	\begin{theorem}
		Pernyataan ekuivalen untuk ideal tak nol dari gelanggang bilangan bulat:
		\begin{enumerate}
			\item $I$ adalah suatu ideal prima
			\item $I = p\mathbb{Z}$ untuk suatu $p$ prima
			\item $I$ adalah ideal maksimal
		\end{enumerate}
	\end{theorem}
	\begin{proof}
		Kita akan buktikan ekuivalensi ini sebagai rantai implikasi (1) $\implies$ (2) $\implies$ (3) $\implies$ (1).\\
		
		Pernyataan (1) $\implies$ (2) adalah teorema \ref{id-prime}. \\
		
		Untuk pernyataan (2) $\implies$ (3), andaikan $I = p\mathbb{Z}$ bukan ideal maksimal, berarti ada ideal $J$ dengan $p\mathbb{Z} \subseteq J$ dan $J \ne \mathbb{Z}$. Ambil $x \in J \setminus p\mathbb{Z}$ (yakni unsur di $J$ yang tidak di $p\mathbb{Z}$), akibatnya haruslah $p$ tidak membagi $x$. Karena $p$ tidak membagi $x$, haruslah GCD($p,x$) = 1. Berarti ada $a,b \in \mathbb{Z}$ sehingga $ap + bx = 1$. Karena $ap \in p\mathbb{Z} \subseteq J$ dan $bx \in J$, haruslah ruas kiri persamaan tersebut ada di $J$. Oleh karena itu, ruas kanannya juga harus di $J$ sehingga $1 \in J$. Akibatnya haruslah $J = \mathbb{Z}$, kontradiksi dengan asumsi bahwa $J \ne \mathbb{Z}$. Kita dapatkan $p\mathbb{Z}$ adalah ideal maksimal.
		\\
		
		Berikutnya untuk (3) $\implies$ (1), misalkan $I$ adalah ideal maksimal. Andaikan $I$ bukan ideal prima, berarti ada $a,b \in \mathbb{Z}$ dengan $ab \in I$ tetapi baik $a$ ataupun $b$ bukan anggota $I$. Karena $I$ ideal maksimal, ideal $a + I$ haruslah $\mathbb{Z}$. Berarti, $1 \in a+I$ sehingga $1 = xa + cy$ untuk suatu $x, y \in \mathbb{Z}$ dan $c \in I$. Dengan cara serupa, kita bisa konstruksi $1 = pb + qr$ untuk suatu $p,r \in \mathbb{Z}$ dan $q \in I$. Mengalikan keduanya, kita dapatkan $1 = (pb+qr)(xa+cy)$. Jabarkan untuk mendapatkan $1 = pbxa + pbcy + qrxa + qrcy$. Perhatikan bahwa karena $\mathbb{Z}$ komutatif, haruslah $pbxa = pabx$. Karena $ab, q, c \in I$, haruslah $1 = pabx + pbcy + qrax + qrcy \in I$. Karena $1 \in I$, haruslah $I = \mathbb{Z}$, kontradiksi dengan pernyataan bahwa $I$ adalah ideal maksimal (karena ideal maksimal haruslah ideal sejati). Dengan demikian, $I$ haruslah merupakan ideal prima.
		\\
		
		Karena rantai implikasinya telah ditunjukkan, ekuivalensi tersebut terbukti.
	\end{proof}
	\textbf{Diskusi.} Tinjau gelanggang $\mathbb{Z} \times \mathbb{Z}$ dengan operasi per komponen. Misalkan $I \subseteq \mathbb{Z} \times \mathbb{Z}$ dengan $I = \{ (a,0) \, | \, a \in \mathbb{Z} \}$. Apakah $I$ adalah ideal prima? Apakah $I$ adalah ideal maksimal?\\
	
	\textbf{Diskusi.} Tunjukkan bahwa $D$ adalah lapangan jika $D$ suatu daerah integral dengan setiap ideal sejati dari $D$ adalah ideal prima. (Hint: Tinjau ideal yang dibangun oleh $a^2$ untuk sebarang $a \in D$)\\
	
	\textbf{Diskusi.} Tinjau ideal tak nol dari suatu gelanggang $R$. Apakah ideal prima selalu merupakan ideal maksimal? Bagaimana dengan sebaliknya, apakah ideal maksimal selalu merupakan ideal prima? Kapankah kesamaan itu terjadi?
	\section{Daerah Euclid}
	Tinjau kembali fungsi nilai mutlak pada gelanggang bilangan bulat. Untuk sebarang $n \in \mathbb{Z}$ kita punyai $|n| = \begin{cases}
	n &\mbox{untuk $n \ge 0$}\\
	-n &\mbox{untuk $n < 0$}
	\end{cases}$\\
	Kita ketahui sifat-sifat dari operasi nilai mutlak tersebut sebagai berikut:
	\begin{enumerate}
		\item $|n| \le |nm|$ untuk setiap $n,m \in \mathbb{Z}$
		\item Misal $n, m \in \mathbb{Z}$ dengan $m \ne 0$. Dari algoritma pembagian, kita dapatkan ada $p,q \in \mathbb{Z}$ sehingga $n = pm + q$ dengan $|q| < |m|$ atau $q = 0$
	\end{enumerate}
	Kata kunci di sini adalah algoritma pembagian (Euclid). Dapatkah kita generalisasi untuk gelanggang yang lebih umum?\\
	
	Misal $D$ suatu daerah integral, definisikan fungsi Euclid $$\phi: D \setminus \{0\} \rightarrow \mathbb{Z}_{\ge 0}$$
	yang memenuhi \begin{enumerate}
		\item $\phi(a) \le \phi(ab)$ untuk setiap $a,b \in D$
		\item Misal $a,b \in D$ dengan $b \ne 0$. Ada $p,q \in D$ sehingga $a = pb + q$ dengan $\phi(q) < \phi(b)$ atau $q = 0$
	\end{enumerate}
	Daerah integral yang dilengkapi dengan fungsi Euclid disebut sebagai \textbf{Daerah Euclid}. Sebagai contoh, $(\mathbb{Z}, | \cdot |)$ dan $(F[x], deg(f))$ adalah daerah Euclid (silahkan anda periksa).\\
	
	Dari pendefinisiannya, jelas bahwa daerah Euclid juga adalah daerah integral. Teorema berikut akan menunjukkan hubungannya dengan daerah ideal utama
	\begin{theorem}
		Jika $D$ adalah daerah Euclid dengan fungsi Euclid $f$, $D$ adalah suatu daerah ideal utama.
	\end{theorem}
	\begin{proof}
		Misal $I$ ideal dari suatu daerah Euclid $D$. Andaikan $I = \{0\}$, jelas $I$ adalah ideal utama. Untuk $I$ lainnya, definisikan himpunan $$T = \{ f(x) \, | \, x\in D \}$$
		Dari prinsip \textit{well-ordering}, $T$ punya unsur terkecil, sebut saja $f(y)$ untuk suatu $y \in I$. Klaim: $I = \langle y \rangle$. Akan kita buktikan klaim tersebut. Ambil $z \in I$. Karena $D$ daerah Euclid, ada $p,q \in D$ sehingga $z = py + q$ dengan $f(q) < f(y)$ atau $q = 0$. Perhatikan bahwa karena $f(y)$ unsur terkecil di $T$, haruslah $q = 0$ sehingga $z = py$. Jadi, $I = \langle y \rangle$. Karena $I$ dibangun oleh satu unsur  untuk sebarang ideal $I$ dari $D$, kita simpulkan bahwa $D$ adalah suatu daerah ideal utama.
	\end{proof}
	Kemudian terkait fungsi Euclid itu sendiri, kita punyai sifat berikut
	\begin{theorem}
		Misal $D$ daerah Euclid dengan fungsi Euclid $f$. Berlaku:
		\begin{enumerate}
			\item $f(1) \le f(a)$ untuk setiap $a \in D \setminus \{0\}$
			\item $f(1) = f(u)$ jika dan hanya jika $u$ unit di $D$
		\end{enumerate}
	\end{theorem}
	\begin{proof}
		Perhatikan bahwa (1) adalah konsekuensi dari pendefinisian fungsi Euclid dengan menulis $a = 1a$ sehingga $f(1) \le f(1a) = f(a)$\\
		
		Kemudian untuk (2), misalkan $f(1) = f(u)$ untuk suatu $u \in D \setminus \{0\}$. Dari algoritma pembagian, kita bisa tuliskan $1 = pu + r$ dengan $f(r) < f(u)$ atau $r = 0$ untuk $p,r \in D \setminus \{0\}$. Andaikan $r \ne 0$, haruslah $f(r) < f(u)$. Padahal $f(u) = f(1)$ sehingga kita dapatkan $f(r) < f(1)$, menyalahi bagian (1) teorema kita. Akibatnya, haruslah $r = 0$ sehingga $u$ adalah unit. \\
		
		Arah sebaliknya, misalkan $u$ adalah unit, berarti ada $u^{-1}$ sehingga $uu^{-1} = 1$. Jadi, $f(u) \le f(uu^{-1}) = f(1)$. Dari bagian (1), haruslah $f(1) \le f(u)$, sehingga kita dapatkan $f(1) = f(u)$.
	\end{proof}
	Dengan dasar pendefinisian daerah Euclid dari algoritma pembagian, kita dapat lakukan pula algoritma Euclid pada daerah Euclid. Misal $(D,f)$ membentuk daerah Euclid dengan $\{a_1, a_2\} \subseteq D$. Dengan algoritma pembagian, tulis $a_1 = p_1 a_1 + a_3$. Andaikan $a_3 \ne 0$, kita punyai $f(a_3) < f(a_2)$. Kita bisa tuliskan lagi $a_2 = p_2 a_3 + a_4$ dengan $f(a_4) < f(a_3)$ dan seterusnya selama $a_n \ne 0$ untuk $n = 1,2,...$. Apa jaminan bahwa proses ini akan berakhir? Perhatikan bahwa nilai-nilai fungsi Euclid tersebut akan membentuk rantai turun $f(a_2) > f(a_3) > ... > 0$. Pada langkah terakhir proses tersebut, kita akan dapatkan
	\begin{equation*}
	\begin{split}
	a_{n-1} & = p_{n-1}a_n + a_{n+1} \\
	a_n & = p_n a_{n+1}
	\end{split}
	\end{equation*}
	Seperti algoritma Euclid pada bilangan bulat, kita akan dapatkan bahwa $a_{n+1}$ adalah GCD($a_1, a_2$). Kita akan buktikan bahwa $a_{n+1}$ tersebut memang benar GCD($a_1, a_2$).
	\begin{theorem}
		Unsur $a_{n+1}$ dalam algoritma Euclid tersebut adalah GCD dari $a_1$ dan $a_2$
	\end{theorem}
	\begin{proof}
		Tinjau unsur $a_i$ pada algoritma Euclid. Akan ditunjukkan bahwa $$\langle a_i, a_{i+1} \rangle = \langle a_{i+1}, a_{i+2} \rangle$$
		Pertama, jelas $a_{i+1} \in \langle a_{i}, a_{i+1} \rangle$ dan $\langle a_{i+1}, a_{i+2} \rangle$. Perhatikan bahwa $a_i = p_i a_{i+1}$, sehingga $\langle a_{i}, a_{i+1} \rangle \subseteq \langle a_{i+1}, a_{i+2} \rangle$.\\
		Dengan cara serupa dapat ditunjukkan bahwa $\langle a_{i+1}, a_{i+2} \rangle \subseteq \langle a_{i}, a_{i+1} \rangle$ sehingga $\langle a_{i}, a_{i+1} \rangle = \langle a_{i+1}, a_{i+2} \rangle$. Dari teorema \ref{generatorIU}, kita tahu $\langle A \rangle = \langle d \rangle$ jika dan hanya jika $d =$ GCD($A$). Berarti haruslah GCD($a_i, a_{i+1}$) = GCD($a_{i+1}, a_{i+2}$). Akibatnya, GCD($a_1, a_2$) = GCD($a_n, a_{n+1}$) = $a_{n+1}$.
	\end{proof}
	\textbf{Diskusi.} Tinjau bilangan bulat Eisenstein $\mathbb{Z}[\omega] = \{a + b\omega \, | \, a,b \in \mathbb{Z} \}$ dengan $\omega = \frac{-1+i\sqrt{3}}{2}$. Apakah ia membentuk daerah Euclid dengan fungsi Euclid $\phi(a+b\omega) = a^2 + b^2 - ab$?
	\section{Daerah Faktorisasi Tunggal}
	Pada bilangan bulat, dalam membicarakan faktorisasi prima suatu bilangan, kita mengenal teorema dasar aritmatika.
	\begin{theorem}[Teorema Dasar Aritmatika]
		Setiap bilangan bulat lebih dari 1 adalah bilangan prima atau merupakan hasil perkalian bilangan prima secara tunggal
	\end{theorem}
	Konsep ini dapat diperluas pula pada struktur abstrak gelanggang. Sebagai contoh, faktorisasi polinomial adalah topik yang banyak dikaji, termasuk keterkaitannya dengan faktorisasi pada bilangan bulat. Lagi-lagi di sini kita akan bicara tentang daerah integral sebagai perluasan dari konsep bilangan bulat. Definisi berikut akan menjadi bahasan kita terkait faktorisasi tunggal pada daerah integral.\\
	
	\textbf{Definisi.} Daerah integral $D$ adalah \textbf{daerah faktorisasi tunggal} (\textit{unique factorization domain}, UFD) jika untuk setiap $x \in D$ yang bukan unit dan bukan nol terdapat unsur-unsur tak-tereduksi $p_1, p_2, ..., p_n \in D$ dan unit $a$ sehingga $$x = ap_1p_2...p_n$$ secara tunggal terhadap permutasi urutan dan unsur sekawan.\\
	
	Perhatikan bahwa sebelumnya kita membahas faktorisasi prima, tetapi pada pendefinisian daerah faktorisasi tunggal, kita gunakan faktorisasi ke unsur-unsur tak-tereduksi. Secara umum, unsur tak-tereduksi berbeda dengan unsur prima. Namun, akan ditunjukkan bahwa pada daerah faktorisasi tunggal, kedua konsep tersebut ekuivalen.
	\begin{theorem}
	Misal $D$ adalah UFD. Unsur $x \in D$ tak-tereduksi jika dan hanya jika $x$ prima.
	\end{theorem}
	\begin{proof}
		($\implies$) Misal $x$ unsur tak tereduksi di $D$. Misal $a,b \in D$ sedemikian  sehingga $x|ab$. Akan ditunjukkan bahwa $x | a$ atau $x|b$. Jika salah satu dari $a,b$ nol atau unit, jelas berlaku. Untuk kasus lainnya, perhatikan bahwa karena $x | ab$, ada $y \in D$ sehingga $ab = xy$. Tinjau kasus untuk $y$ bukan unit. Kita bisa lakukan faktorisasi $a = a_1 a_2 ... a_k$, $b = b_1 b_2 ... b_l$, dan $y = y_1 y_2 ... y_m$, yaitu sebagai hasil kali faktor-faktor tak-tereduksi. Kita dapatkan $$a_1 a_2 ... a_k b_1 b_2 ... b_l = xy_1 y_2 ... y_m$$Karena $D$ adalah UFD, haruslah $x$ sekawan dengan $a_i$ untuk suatu $i \in [1,k]$ atau $x$ sekawan dengan $b_j$ untuk suatu $j \in [1,l]$. Kita dapatkan haruslah $x | a$ atau $x | b$ sehingga $x$ prima.\\
		
		($\impliedby$) Misal $p \in D$ prima dengan $p = ab$ untuk suatu $a,b \in D$. Akan ditunjukkan bahwa haruslah $a$ unit atau $b$ unit. Karena $p = ab$, jelas $p | ab$, sehingga haruslah $p | a$ atau $p | b$. Tanpa mengurangi keumuman, misalkan $p | a$. Karena $p = ab$, kita punyai juga $a | p$ sehingga $a$ sekawan dengan $p$. Berarti $p$ dan $a$ dapat ditulis sebagai $p = ak$ untuk suatu $k$ unit. Akibatnya, $ak = ab$. Karena UFD adalah daerah integral, kita dapat lakukan pencoretan sehingga $k = b$. Kita dapatkan $b$ adalah unit. Jadi, $p$ tak-tereduksi.
	\end{proof}
	Karena ekuivalensi tersebut telah ditunjukkan, kita akan menganggap dua istilah tersebut sama pada pembahasan tentang UFD. Berikutnya kita akan bahas hubungan antara daerah faktorisasi tunggal dengan struktur lain yang terkait. Pertama, jelas bahwa UFD pastilah merupakan suatu daerah integral. Kemudian, akan ditunjukkan bahwa setiap daerah ideal utama adalah UFD.
	\begin{theorem}
		Daerah ideal utama adalah suatu daerah faktorisasi tunggal
	\end{theorem}
	\begin{proof}
		Misalkan $D$ adalah suatu daerah ideal utama dan $x \in D$. Andaikan $x$ prima, kita bisa tulis $x = ac$ untuk suatu unit $a$ dan $c$ prima. Andaikan tidak, tulis $x = b_1 c$ dengan $b_1, c$ bukan unit dan bukan prima. Kita bisa lakukan terus prosesnya hingga didapat $x$ sebagai perkalian unsur-unsur prima dan unit. Akan ditunjukkan bahwa proses ini akan senantiasa berhenti. Tinjau teorema \textit{ascending chain condition}. Misal $b_1, b_2, ...$ bukan unsur prima dengan $b_j | b_i$ jika dan hanya jika $i \le j$. Akibatnya, rantai ideal $$\langle b_1 \rangle \subseteq \langle b_2 \rangle \subseteq ...$$punya ideal terbesar. Jadi, ada $k$ sehingga $\langle b_k \rangle = \langle b_{k+j} \rangle$ untuk setiap $j \ge 0$. Akibatnya, $b_{k+j}$ tak bisa lagi ditulis sebagai hasil kali 2 bilangan sehingga proses pemfaktoran tersebut akan berhenti. Kita dapatkan $x = ap_1p_2...p_n$ dengan $a$ unit dan $p_i$ prima.\\
		
		Berikutnya, perlu diperiksa bahwa faktorisasi tersebut tunggal terhadap unsur sekawan dan permutasi. Misalkan $x = ap_1p_2...p_n$ dan $x = bq_1q_2...q_m$. Akan ditunjukkan bahwa $m = n$, $a$ sekawan $b$, dan ada permutasi $\sigma: [1,n] \rightarrow [1,n]$ sehingga $p_i$ sekawan dengan $q_{\sigma(i)}$.\\
		
		Perhatikan bahwa karena $a,b$ keduanya unit, haruslah $p_1p_2...p_n$ sekawan dengan $q_1q_2...q_m$. Akibatnya, untuk suatu $i,j$, $p_i | q_j$ secara tunggal. Kita bisa tulis \begin{equation*}
		\begin{split}
			x & = b u_1 p_1 u_2 p_2 ... u_m p_m\\
			& = u p_1 p_2 ... p_m\\
			& = ap_1p_2...p_n
		\end{split}
		\end{equation*}
		untuk $u$ suatu unit. Jadi haruslah $m = n$. Berarti $a$ sekawan dengan $b$ dan ada permutasi $\sigma: [1,n] \rightarrow [1,n]$ sehingga $p_i$ sekawan $q_{\sigma({i})}$. Jadi, $x$ terfaktorisasi secara tunggal. Kita simpulkan bahwa $D$ adalah suatu daerah faktorisasi tunggal.
	\end{proof}
	\textbf{Diskusi.} Periksa untuk $n \in \mathbb{N}$ berapa sajakah $\mathbb{Z}[\sqrt{-n}]$ merupakan daerah faktorisasi tunggal. Apakah ia juga daerah Euclid untuk fungsi Euclid $\rho_n (a+b\sqrt{-n}) = a^2 + nb^2$? Tinjau $n = 3$, coba cari contoh unsur prima di $\mathbb{Z}[\sqrt{-3}]$
	\section{Penutup}
	Pada bab ini kita telah mengkaji beberapa struktur dengan properti-propertinya. Berikut adalah diagram yang menyatakan hubungan antara struktur-struktur tersebut (panah dari $x$ ke $y$ menyatakan $x$ adalah $y$):\\
	
	\begin{tikzcd}
	& Gelanggang                                                         &                                                 & Lapangan \arrow[lldd] \\
	& Gel. \ Komutatif \arrow[u]                                         &                                                 &                       \\
	& Daerah \ Integral \arrow[u] \arrow[rruu, "Berhingga"', bend right] &                                                 &                       \\
	& UFD \arrow[u]                                                      & PID \arrow[l] & Daerah \ Euclid \arrow[l]  &                      
	\end{tikzcd}\\
	
	Masih banyak struktur lainnya dengan beragam motivasi terkait pendefinisiannya yang tidak dibahas pada materi struktur bilangan bulat. Pembahasan terkait grup, kuosien, dan homomorfisma juga tidak dibahas di sini. Bagi pembaca yang tertarik belajar lebih jauh, penulis menyarankan untuk membaca lebih lanjut pustaka utama yang menjadi referensi kuliah ini.
	

\appendix
\chapter{Grup}
	Di apendiks ini kita hanya akan sedikit berkenalan dengan konsep grup dalam aljabar abstrak. Jika pembaca ingin belajar lebih jauh, silahkan membaca buku referensi utama untuk kuliah ini (atau kuliah struktur aljabar). \\
	
	Tinjau sistem matematika $(S,\oplus)$ untuk suatu himpunan tak kosong $S$ dan operasi $\oplus: S \times S \rightarrow S$. Sistem matematika $(S, \oplus)$ (atau tulis sebagai $S$ saja apabila operasi yang dimaksud jelas) kita sebut sebagai \textbf{grup} apabila:
	\begin{enumerate}
		\item Terdapat unsur identitas $e \in S$ sehingga untuk sebarang $a \in S$ berlaku $a \oplus e = e \oplus a = a$ (terkadang juga dinotasikan sebagai nol)
		\item Untuk sebarang $a \in S$ terdapat unsur invers $-a \in S$ (atau seringkali juga dinotasikan sebagai $a^{-1}$) sehingga $a \oplus (-a) = -a \oplus a = e$
		\item Untuk sebarang $a,b,c \in S$ berlaku $a \oplus (b \oplus c) = (a \oplus b) \oplus c$
	\end{enumerate}
	Apabila $a \oplus b = b \oplus a$ untuk setiap $a,b \in S$, kita katakan $S$ sebagai \textbf{grup komutatif}. Tentunya karena $\oplus$ suatu operasi, $S$ haruslah tertutup atas $\oplus$ serta operasinya harus terdefinisi dengan baik (tidak bergantung pada wakil kelas). Dengan kata lain, untuk $a,b,a',b' \in S$ dengan $a = a'$ dan $b = b'$ haruslah berlaku $a \oplus b = a' \oplus b'$ (kasus seperti ini akan kita temui ketika kita meninjau grup yang anggotanya adalah kelas-kelas ekuivalen, misalnya $(\mathbb{Z}_2, +)$ yang berisi kelas ekuivalen $[0]_2$ dan $[1]_2$). Dapat kita periksa bahwa unsur identitas tunggal di grup dan unsur invers bagi suatu unsur juga tunggal.
	\begin{theorem}[Ketunggalan Identitas]
	Misalkan $d,e$ unsur di grup $(S,*)$ dan untuk setiap $a \in S$ berlaku $e * a = a * e = a$ dan $d * a = a * d = a$. Haruslah berlaku $e = d$.
	\end{theorem}
	\begin{proof}
		 Perhatikan bahwa \begin{equation*}
		\begin{split}
		e * a &= a * e \\
		&= a\\
		&= d * a\\
		\end{split}`
		\end{equation*}	
		Sehingga dengan mengoperasikan $a^{-1}$ di kedua ruas dari kanan, kita dapatkan $e = d$. Jadi, haruslah unsur identitas tunggal
	\end{proof}
	\begin{theorem}[Ketunggalan Invers]
	Misalkan $(S,*)$ suatu grup dan $a,b,c \in S$ sedemikian sehingga $a * b = b * a = e$ dan $a * c = c * a = e$ untuk $e$ unsur identitas di $S$. Haruslah berlaku $b = c$.
	\end{theorem}
	\begin{proof}
		Tinjau unsur $b$. Karena $e$ identitas, haruslah $b = b * e$. Kemudian, perhatikan bahwa
		\begin{equation*}
		\begin{split}
		b * e & = b * (a * c)\\
		& = (b * a) * c\\
		& = e * c\\
		& = c
		\end{split}
		\end{equation*}
		Jadi $b = c$ dan kita selesai.
	\end{proof}
	Untuk mempermudah penulisan, umumnya grup $(S,*)$ hanya akan ditulis sebagai grup $S$ saja dan operasinya tidak ditulis. Sebagai contoh, operasi $a * b * a^{-1} * b^{-1}$ umumnya hanya akan ditulis sebagai $aba^{-1}b^{-1}$ saja. Terkadang juga ketika kita bekerja dengan lebih dari satu grup (misalnya ketika kita bekerja dengan homomorfisma grup dari grup $G$ ke grup $H$), akan berguna untuk menandai identitas $G$ sebagai $e_G$ dan identitas $H$ sebagai $e_H$. Pelabelan ini dapat dihilangkan jika dirasa cukup jelas grupnya.
	\\
	
	\textbf{Diskusi.} Periksa apakah operasi + pada grup $(Z_n, +)$ dengan $+$ adalah penjumlahan modulo terdefinisi dengan baik\\
	
	\textbf{Diskusi.} Tunjukkan bahwa jika suatu unsur adalah identitas satu sisi dari suatu grup, haruslah ia unsur identitas di grup tersebut. Yaitu, tunjukkan bahwa jika untuk suatu grup $G$ dengan $a,b \in G$ dan $ab = b$ haruslah berlaku $ba = b$ dan $a = e_G$\\
	
	\textbf{Diskusi.} Misalkan $(G,\star)$ dan $(H,\circ)$ grup dan $f: G \rightarrow H$ suatu pemetaan yang mengawetkan operasi, yaitu $f(a \star b) = f(a) \circ f(b)$ untuk setiap $a,b \in G$. Periksa apakah:
	\begin{enumerate}
		\item $f(e_G) = e_H$
		\item Untuk sebarang $a \in G$ berlaku $f(a^{-1}) = {f(a)}^{-1}$
		\item Peta($f$)dan Inti($f$) adalah suatu grup
		\item Untuk sebarang $a,b \in G$ haruslah $a \star b \star a^{-1} \star b^{-1} \in$ Inti$(f)$
	\end{enumerate}
	Pemetaan yang mengawetkan operasi tersebut kita kenal sebagai \textbf{homomorfisma} dari grup $G$ ke grup $H$

\end{document}