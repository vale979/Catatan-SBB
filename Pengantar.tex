\chapter{Pengantar Untuk Versi Pertama}
	Catatan ini dibuat sebagai penunjang dan referensi tambahan ringkas untuk materi kuliah struktur bilangan bulat. Sebagian besar isi dari catatan ini adalah digitalisasi dari catatan penulis saat mengikuti kuliah struktur bilangan bulat di prodi Matematika ITB pada semester genap tahun akademik 2018-2019. Beberapa tambahan yang relevan sebagian besar diambil dari buku \textit{Aljabar} karya Achmad Ariffin, buku \textit{Abstract Algebra} karya I.N. Herstein, atau hasil corat-coret penulis. Kedua buku tersebut merupakan buku teks yang dijadikan referensi pada kuliah struktur bilangan bulat. Sebagai apresiasi saya terhadap prinsip kemerdekaan atas ilmu pengetahuan, catatan ini bebas untuk disebarluaskan, dicetak, dan digunakan untuk tujuan non-komersil.
	\\
	
	Sesuai namanya, catatan ini hendaknya hanya digunakan sebagai catatan untuk \textit{review} materi terkait. Catatan ini tidak ditujukan sebagai pengganti buku teks, apalagi referensi satu-satunya untuk mempelajari struktur bilangan bulat. Sebaiknya pembaca terlebih dahulu membaca buku teks yang menjadi referensi utama sebelum membaca catatan ini. Karena kuliah ini merupakan kuliah tingkat dua, penulis mengasumsikan pembaca sudah familiar dengan notasi-notasi yang umum digunakan pada operasi himpunan dan pemetaan (notasi penting lainnya akan diberikan definisinya). Penulis juga mengasumsikan pembaca telah familiar dengan metode penulisan bukti yang diajarkan pada kuliah pengantar matematika. Istilah grup sengaja penulis gunakan meskipun tidak masuk dalam kajian kuliah struktur bilangan bulat untuk mempermudah pendefinisian beberapa struktur. Sekilas bahasan tentang grup dapat diperiksa pada bagian lampiran.
	\\
	
	Meskipun ini bukanlah buku teks, penulis berharap metode membaca catatan ini serupa dengan membaca buku teks. Pembaca harus aktif memverifikasi hasil yang didapat pada catatan ini dengan setidaknya selembar kertas dan sebuah pulpen serta mempertanyakan motivasi dibalik pendefinisian suatu ide (umumnya jawabannya akan anda dapatkan setelah membaca topik bahasan terkait ide tersebut). Beberapa soal (sebagian diambil dari soal yang dibahas di kelas) disisipkan penulis untuk menambah gambaran sekaligus melatih konsep yang dibahas.
	\\
	
	\textit{Feedback} untuk catatan ini dapat diberikan dengan menghubungi penulis via surel di valerianmp@valmc2.com atau melalui sosial media lainnya. Jika dirasa perlu, versi revisi berikutnya akan dirilis untuk memperbaiki catatan ini. Sebagai akhir dari pengantar ini, tak lupa penulis mengucapkan terima kasih kepada Bu Dellavitha sebagai dosen pengampu kuliah struktur bilangan bulat beserta teman-teman sekelas dalam kuliah struktur bilangan bulat pada semester genap 2018-2019. Penulis berharap catatan ini dapat dimanfaatkan sebaik-baiknya, entah dalam membantu pada kuliah struktur bilangan bulat atau hal lainnya yang relevan.
	\\
	\begin{flushright}	
	Untuk umat manusia yang lebih baik,\\
	\hfill \break
	Valerian Mahdi Pratama
	\end{flushright}
	